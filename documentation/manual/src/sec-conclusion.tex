\FloatBarrier
\section{Lessons Learned and Future Directions} 
\label{sec:lessons-learned}

The DDAD workbench is an open source project hosted on Github. This is important
because workbench systems tend to garner little use outside of their creators
and they tend to fall by the wayside as time goes on. Part of our contribution
is creating an open and extensible platform for people to build upon. This can
include not just active development by a core team of researchers but also
contributions by students, professional programmers, or hobbyist developers. The
repository is full of code not listed here, including work on Voronoi diagrams
and Boolean operations. While making a project open source and easy to access is
not sufficient to keep it from falling into disuse, it does significantly lower
the barrier to potential contributors.

The Wiki contains documentation concerning how to contribute to the project,
including coding guidelines, compilation instructions, how to deploy
a new workbench, etc. A snapshot of many of these pages is included in the
appendix. This document and the Wiki are also open source and can be maintained
and updated by the development community.

We maintain an active list of issues on the Github repository that defines
current limitations of the system and future directions that we would like to
explore. In the future, the most direct and interesting avenue for exploration
is implementing `order independent transparency` and a 3D convex hull algorithm.
This will allow users to empirically test out possible algorithms for the 3D
integer hull. Other interesting directions include visualizing unimodular
transformations, which skew the integer lattice (e.g. drawing multiple grid
overlays.)


% Implementing a geometric algorithm workbench is a challenging task with a rich
% set of problems encompassing a variety of disciplines. We converged on an
% interesting events design in which the user annotates observable objects to
% signal changes in their visual state. The user specifies visual state on points,
% line segments, and triangles, determines delay length between animation
% sequences, and directs the viewing camera while the animation runs.
% 	
% Many opportunities for improvement and further exploration exist. First,
% implementing existing plans of debugging controls affords an immediate
% improvement in the tool's usefulness. Second, allowing the presenter to specify
% sounds to accompany interesting events provides another means of conveying
% information. Finally, creating better abstractions of common data structures and
% investigating integration with existing debuggers reduces system invasiveness
% for the implementer.