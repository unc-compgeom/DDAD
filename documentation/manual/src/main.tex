%==============================================================================
% @author Clinton Freeman <freeman@cs.unc.edu>
% @date 09/19/2013
%==============================================================================

\documentclass{article}

\usepackage{fullpage}
\usepackage{amsfonts}
\usepackage{amssymb}
\usepackage{amsmath}
\usepackage{amsthm}
\usepackage[usenames,dvipsnames,svgnames,table]{xcolor}
\usepackage[labelfont=bf]{caption}
\usepackage{url}
\usepackage[linktocpage=true]{hyperref}
\usepackage{algorithm}
\usepackage{algpseudocode}
\usepackage{esvect}
\usepackage{longtable}
\usepackage{enumitem}
\usepackage{mdframed}
\usepackage{titlesec}
\usepackage{etoolbox}
\usepackage{float}
\usepackage{fancybox}
\usepackage{graphicx}
\usepackage{tabularx}
\usepackage{wrapfig}
\usepackage{booktabs}
\usepackage{tikz}
\usepackage{calc}
\usepackage{placeins}
\usepackage{xcolor}
\usepackage{rotating}

\definecolor{algback}{rgb}{0.95, 0.95, 0.95}

\usetikzlibrary{calc, decorations.pathmorphing, shapes, arrows.new}

\hypersetup {
    colorlinks = true,
    linkcolor = {blue},
    citecolor = {blue},
    urlcolor = {blue}
}

\definecolor{keyword}{HTML}{445588}

\newtheorem{theorem}{Theorem}
\newtheorem{corollary}{Corollary}
\newtheorem{lemma}{Lemma}
\newtheorem{invariant}{Invariant}
\theoremstyle{definition}
\newtheorem{definition}{Definition}
 
\renewcommand{\algorithmicrequire}{\textbf{Input:}}
\renewcommand{\algorithmicensure}{\textbf{Output:}}

\newcommand{\floor}[1]{\left\lfloor #1 \right\rfloor}
\newcommand{\ceiling}[1]{\left\lceil #1 \right\rceil}

 
\newcommand{\tightoverset}[2]{
    \mathop{#2}\limits^{\vbox to -.8ex{\kern-0.7ex\hbox{$#1$}\vss}}
}

\newcommand{\tikzoverset}[2]{%
  \tikz[baseline=(X.base),inner sep=0pt,outer sep=0pt]{%
    \node[inner sep=0pt,outer sep=0pt] (X) {$#2$}; 
    \node[yshift=1pt] at (X.north) {$#1$};
}}
 
\newlength{\mOLineLength}
\newcommand{\mOLine}[1]{
    \overset{
        \setlength{\mOLineLength}{\widthof{#1}}
        \begin{tikzpicture}
            \draw [-to new, arrow head = 1.6pt, line width=0.5pt] (0, 0) --
            +(0.8\mOLineLength, 0);
        \end{tikzpicture}
    }{#1}
}

\newlength{\mRayLength}
\newcommand{\mRay}[1]{
    \tightoverset{
    \setlength{\mRayLength}{\widthof{#1}}
    \begin{tikzpicture}
        \draw [* new-to new, arrow head = 1.6pt, line width=0.5pt]
        (0, 0) -- +(0.8\mRayLength, 0);
        %\filldraw (0,0) circle (0.6pt);
        %\draw [-angle 45 new, arrow head = 2.75pt, line width=0.4pt] (0, 0) --
        %+(0.8\mOLineLength, 0);
    \end{tikzpicture}
    }{\mathrm{#1}}
}

\newlength{\mSegLength}
\newcommand{\mSeg}[1]{
    \tightoverset{
    \setlength{\mSegLength}{\widthof{#1}}
    \begin{tikzpicture}
        \draw [* new-* new, arrow head = 1.6pt] (0, 0) -- +(0.8\mSegLength,
        0);
    \end{tikzpicture}
    }{\mathrm{#1}}
}

\newcommand{\mR}[1]{\mathbb{R}^{#1}}
\newcommand{\mZ}[1]{\mathbb{Z}^{#1}}
\newcommand{\mPoint}[1]{\mathrm{#1}}
\newcommand{\mPt}[1]{\mPoint{#1}}
\newcommand{\mVector}[1]{\mathbf{#1}}
\newcommand{\mVc}[1]{\mVector{#1}}
\newcommand{\mULine}[1]{\tightoverset{\leftrightarrow}{\mathrm{#1}}}
\newcommand{\mMatrix}[1]{\mathbf{#1}}
\newcommand{\mMt}[1]{\mMatrix{#1}}
\newcommand{\mPlane}[1]{\mathrm{#1}}
\newcommand{\mPolygon}[1]{\mathrm{#1}}
\newcommand{\mPolytope}[1]{\mathrm{#1}}
\newcommand{\mSet}[1]{\mathrm{#1}}
\newcommand{\mField}[1]{\mathrm{#1}}
\newcommand{\mCompound}[1]{\mathrm{#1}}
\newcommand{\mIntHull}[1]{\text{IH}(#1)}
\newcommand{\mBoundary}[1]{\partial #1}
\newcommand{\mTernary}[3]{\left( #1 \right) \text{?} #2 \text{:} #3}

\long\def\ignore#1{\relax}

\def\tomath#1{\relax\ifmmode#1\else$#1$\fi}

\def\th{{\rm th}}
\def\inv{^{-1}}
\def\degrees{\tomath{{}^\circ}}
\let\degree=\degrees
\def\lf{\left}\def\rt{\right}

\def\rangeone#1#2{\rangethree{#1}12{#2}}
\def\rangetwo#1#2#3{\tomath{{#1}_{#2},\ldots,{#1}_{#3}}}
\def\rangethree#1#2#3#4{\tomath{{#1}_{#2},{#1}_{#3},\ldots,{#1}_{#4}}}
\let\range=\rangetwo
\def\setone#1#2{\setthree{#1}12{#2}}
\def\settwo#1#2#3{\tomath{\{{#1}_{#2},\ldots,{#1}_{#3}\}}}
\def\setthree#1#2#3#4{\tomath{\{{#1}_{#2},{#1}_{#3},\ldots,{#1}_{#4}\}}}

\def\oldO{
\def\Om(##1){{\tomath{\Omega(##1)}}}
\def\Th(##1){{\tomath{\Theta(##1)}}}
\def\Ologn{\O(\log n)}
\def\Onlogn{\O(n\log n)}
\def\O(##1){{\tomath{O(##1)}}}
\def\On##1{\O(n^{##1})}
}

\def\Om#1{{\tomath{\Omega(#1)}}}
\def\Th#1{{\tomath{\Theta(#1)}}}
\def\Ologn{\O{\log n}}
\def\Onlogn{\O{n\log n}}
\def\O#1{{\tomath{O(#1)}}}
\def\On#1{\O{n^{#1}}}


\def\Case#1{\noindent {\bf Case #1:\/ }}
\def\NP{{\sl NP}}
\def\R{\tomath{{\cal R}}}

\def\etal{{et al.{}}}
\def\fourldots{\mathinner{\ldotp\ldotp\ldotp\ldotp}}
\def\fourdots{\relax\ifmmode
    \fourldots\else$\mathsurround=0pt \fourldots\,$\fi
    \spacefactor=3000}


\def\nopar#1\par{}
\def\slw #1 {{\sl #1 }}
\def\itw #1 {{\it #1 }}
\def\ttw #1 {{\tt #1 }}
\def\scw #1 {{\sc #1 }}
\def\bfw #1 {{\bf #1 }}
\let\bw=\bfw
\def\calw #1 {\tomath{{\cal #1}} }
\def\calv#1{\tomath{{\cal #1}}}

\def\slug{\vrule height 4pt depth 0pt width 4pt}

\def\joinrel{\mathrel{\mkern-4mu}}% fix longrightarrow et c.
\def\ray#1{\hbox{\vbox{\offinterlineskip\setbox0\hbox{$#1$}
    \hbox to \wd0{\hss$\rightharpoonup$\hss}\vskip-1.0pt\box0}}}
\def\lin#1{\hbox{\vbox{\offinterlineskip\setbox0\hbox{$#1$}
    \hbox to \wd0{\hss$\leftrightarrow$\hss}\vskip-1.0pt\box0}}}
\def\seg#1{\tomath{\overline{#1}}}

\def\paper{paper}% for changing papers to thesis chapters

\long\def\comm#1{\ignorespaces}
\def\comments{\long\def\comm##1{\message{COMMENT: ##1}{\bf(( ##1 ))}}}

%\hyphenation{half-space}

\def\raggedcenter{\advance\leftskip by 0pt plus 40em\rightskip=\leftskip
   \parfillskip=0pt \spaceskip=.3333em \xspaceskip=.5em
   \pretolerance=9999 \tolerance=9999
   \hyphenpenalty=9999 \exhyphenpenalty=9999 }

\long\def\tthdump#1{#1} % Do nothing. The following are not done for TtH.
\tthdump{%
%\input Gmacro2.tex
}

\newcommand\quelle[1]{{%
      \unskip\nobreak\hfil\penalty50
      \hskip2em\hbox{}\nobreak\hfil#1%
      \parfillskip=0pt \finalhyphendemerits=0 \par}}


\newcounter{proofc}
\renewcommand\theproofc{\arabic{proofc}. }
\DeclareRobustCommand\stepproofc{\refstepcounter{proofc}\theproofc}
\newenvironment{twoproof}{\noindent \emph{Proof.} \\[0.5em]
\tabular{@{\stepproofc}p{2.5in}p{3.25in}}}{\endtabular \\
\quelle{$\square$}\setcounter{proofc}{0}}

\newcounter{sarrow}
\newcommand\xrsquigarrow[1]{%
\stepcounter{sarrow}%
\mathrel{\begin{tikzpicture}[baseline= {( $ (current bounding box.south) + (0,-0.5ex) $ )}]
\node[inner sep=.5ex] (\thesarrow) {$\scriptstyle #1$};
\path[draw,<-,decorate,
  decoration={zigzag,amplitude=0.7pt,segment length=1.2mm,pre=lineto,pre length=4pt}] 
    (\thesarrow.south east) -- (\thesarrow.south west);
\end{tikzpicture}}%
}


\definecolor{answerColor}{RGB}{245, 245, 245}
\definecolor{titleColor}{RGB}{225, 225, 225}
\mdfdefinestyle{answer} {
    backgroundcolor = answerColor,
    linecolor = answerColor,
    skipabove = 8pt,
    skipbelow = 8pt,
    splittopskip = 22pt,
    splitbottomskip = \baselineskip,
    innertopmargin = 0.75em,
    innerbottommargin = 0.75em,
    innerrightmargin = 0.75em,
    innerleftmargin = 0.75em,
    frametitlealignment=\center,
    frametitlebackgroundcolor=titleColor
}

\definecolor{cBack}{rgb}{0.95, 0.95, 0.95}
\usepackage{listings}
\lstset {
	basicstyle = \footnotesize\ttfamily,
	%numbers = left,
	numberstyle = \tiny,
	%stepnumber = 2,
	numbersep = 5pt,
	tabsize = 2,
	extendedchars = true,
	breaklines = true,
	keywordstyle = \color{red},
	frame = b,
	stringstyle = \color{white}\ttfamily,
	showspaces = false,
	showtabs = false,
	xleftmargin = 8pt,
	framexleftmargin = 8pt,
	framexrightmargin = 5pt,
	framexbottommargin = 4pt,
	showstringspaces = false
}

\lstloadlanguages {
	C++
}

\DeclareCaptionFont{white}{\color{white}}
\DeclareCaptionFormat{listing}{\colorbox[cmyk]{0.43, 0.35, 0.35,0.01}{\parbox{\textwidth}{\hspace{8pt}#1#2#3}}}
\captionsetup[lstlisting]{format=listing,labelfont=white,textfont=white, singlelinecheck=false, margin=0pt, font={sf,bf,footnotesize}}

\def\tomath#1{\relax\ifmmode#1\else$#1$\fi}
\def\opName#1{\hbox{\tt{\textsc{#1}}}}
\def\op#1(#2){\tomath{\opName{#1}(#2)}}

\def\deg(#1){\tomath{\mathord{ \large \textcircled{\small \tomath{#1}} \normalsize }}}

\title{A Geometric Workbench for Degree-Driven Algorithm Design}
\author{
	Clinton Freeman \\
	\texttt{freeman@cs.unc.edu}
	\and
	Jack Snoeyink \\
    \texttt{snoeyink@cs.unc.edu}
}

\begin{document}

\maketitle

%==============================================================================

\begin{abstract}
Millman built a C++ library (DDAD) to facilitate implementing algorithms with
low degree predicates. Our workbench extends DDAD with a visual event system and
provides a standalone GUI that can generate input data and render algorithm
execution.
\end{abstract}

\section{Introduction}

The executions of algorithms on geometric data would seem to be presented most
naturally as animations. But what should be animated? When one looks closely
at a geometric algorithm, one finds different interpretations of what is being
done: points and lines may be compared by computing vector dot and cross
products, which involve arithmetic computation on the coordinate values of
points. An implementer or a presenter of an algorithm may be interested in
different levels of abstraction.

An algorithm \emph{implementer} must correct programming errors and handle
degenerate situations correctly. Traditional debuggers provide only textual or
numerical representations of geometric data structures, and generating
degenerate geometric input is a nontrivial task for which there is often little
assistance. 

An algorithm \emph{presenter} must convey their ideas to audiences of
researchers and students. Many presenters use static depictions and verbally
explain algorithm mechanics, a type of presentation that does not fully capture
the dynamic nature of geometric algorithms.

\begin{figure}[htb]
	\centering
	\includegraphics[width=\textwidth]{figures/currentstate-2}
	\caption{Our workbench animates incremental terrain mesh generation.} 
	\label{fig:currentstate}
\end{figure}

A \emph{geometric workbench} aids algorithm implementers and presenters by
providing facilities to dynamically visualize geometric algorithms. Implementers
can visually inspect geometric relationships and properties of data structures,
quickly recognizing erroneous computations. Presenters can produce
animations of their algorithms, more clearly conveying essential
ideas to their audience. Both types of geometers can interactively control the
flow of execution and easily generate degenerate input data.

\emph{Degree-driven algorithm design} encourages robust geometric computing by
attempting to minimize an algorithm's arithmetic precision with its running time
and space~\cite{millman2012degree}. Millman built a C++ library (DDAD) to
facilitate implementing algorithms with low-degree predicates. Our workbench
extends DDAD with a visual event system and provides a standalone GUI that can
generate input data and render algorithm execution.
Figure~\ref{fig:currentstate} captures the workbench rendering a terrain mesh
built by incremental delaunay triangulation.

This paper explains the benefits of using the DDAD workbench as a platform for
implementing and presenting degree-driven geometric algorithms. In
section~\ref{sec:taxonomy-previousworks-desiderata}, we define a taxonomy for
workbench systems and use it to place our workbench in the context of previous
works. We study Melkman's algorithm in section~\ref{sec:case-melkman} to show
how to leverage existing DDAD types to implement a simple 2-dimensional
algorithm. The inner workings of these types are revealed in
section~\ref{sec:workbench-architecture}. We conclude by studying incremental
Delaunay triangulation to show how to implement a complex 3-dimensional
algorithm.

% This paper provides a guided introduction to the inner workings of the DDAD
% workbench. It is intended for people interested in degree-driven algorithm
% design and who would like to use the DDAD library to get started implementing
% algorithms. We start out by explaining

%[todo: paragraph overviewing the paper]

% Section 2 reviews previous work. Section 3 recalls a general framework for
% designing software visualization systems. Section 4 uses the framework to
% explain our workbench desiderata. Section 5 provides an architectural overview
% of the workbench and section 6 runs through a case study of animating a convex
% hull algorithm. Section 7 concludes by explaining how current workbench
% architecture supports our desiderata and identifies areas in which it can
% improve.

 

\section{General Workbench Desiderata}

From their thorough review of geometric workbench systems, Dobkin and Hausner
extracted a set of design decisions that must be made by new
systems~\cite{hausner1999animation}.
However, a geometric workbench is a specific type of \emph{software
visualization} (SV) system. The SV literature contains several well-developed
taxonomies that are designed to evaluate SV systems~\cite{diehl2007software}. In
particular, Price \emph{et al.} outline six major categories in which they judge
SV systems: scope, content, form, method, interaction, and
effectiveness~\cite{price1993principled}. We found that this detailed taxonomy
provided a more structured way of organizing the desiderata of new workbench
systems.

This section recalls the taxonomy for evaulating SV systems and relates each
category to the specific case of workbench systems. The taxonomy is a hierarchy
of categories, each of which may be described in terms of a question about the
system. This makes the task of describing our specific desiderata quite easy
because we may simply answer each of these questions in turn. 


\subsection{Scope}

What is the range of programs that the SV system may take as input for
visualization?

\begin{enumerate}
  \item Generality. Can the system handle a generalized range of programs or
  does it display a fixed set of examples?
  \begin{enumerate}
    \item Hardware. What hardware does it run on?
    \item Operating System. What operating system is required to run it?
    \item Language. What programming language must user programs be written
    in?
    \begin{enumerate}
      \item Concurrency. If the programming language is capable of
      concurrency, can the SV system visualize the concurrent aspects?
    \end{enumerate}
    \item Applications. What are the restrictions on the kinds of user
    programs that can be visualized?
    \begin{enumerate}
      \item Specialty. What kinds of programs is it particularly good at
      visualizing (as opposed to simply capable of visualizing)?
    \end{enumerate}
  \end{enumerate}
  \item Scalability. To what degree does the system scale up to handle large
  examples?
  \begin{enumerate}
    \item Program. What is the largest program it can handle?
    \item Data Sets. What is the largest input data set it can handle?
  \end{enumerate}
\end{enumerate}


\begin{enumerate}
  \item Scope. What is the range of programs that the SV system may take as
  input for visualization?
  \begin{enumerate}
    \item Generality. Can the system handle a generalized range of programs or
    does it display a fixed set of examples?
    \begin{enumerate}
      \item Hardware. What hardware does it run on?
      \item Operating System. What operating system is required to run it?
      \item Language. What programming language must user programs be written
      in?
      \begin{enumerate}
        \item Concurrency. If the programming language is capable of
        concurrency, can the SV system visualize the concurrent aspects?
      \end{enumerate}
      \item Applications. What are the restrictions on the kinds of user
      programs that can be visualized?
      \begin{enumerate}
        \item Specialty. What kinds of programs is it particularly good at
        visualizing (as opposed to simply capable of visualizing)?
      \end{enumerate}
    \end{enumerate}
    \item Scalability. To what degree does the system scale up to handle large
    examples?
    \begin{enumerate}
      \item Program. What is the largest program it can handle?
      \item Data Sets. What is the largest input data set it can handle?
    \end{enumerate}
  \end{enumerate}
  \item Content. What subset of information about the software is visualized by
  the SV system?
  \begin{enumerate}
    \item Program. To what degree does the system visualize the actual
    implemented program?
    \begin{enumerate}
      \item Code. To what degree does the system visualize the instructions in
      the program source code?
      \begin{enumerate}
        \item Control Flow. To what degree does the system visualize the flow of
        control in the program source code?
      \end{enumerate}
      \item Data. To what degree does the system visualize the data structures
      in the program source code?
      \begin{enumerate}
        \item Data Flow. To what degree does the system visualize the flow of
        data in the program source code?
      \end{enumerate}
    \end{enumerate}
    \item Algorithm. To what degree does the system visualize the high-level
    algorithm behind the software?
    \begin{enumerate}
      \item Instructions. To what degree does the system visualize the
      instructions in the algorithm?
      \begin{enumerate}
        \item Control Flow. To what degree does the system visualize the flow of
        control in the algorithm instructions?
      \end{enumerate}
      \item Data. To what degree does the system visualize the data structures
      in the algorithm?
      \begin{enumerate}
        \item Data Flow. To what degree does the system visualize the flow of
        data in the algorithm?
      \end{enumerate}
    \end{enumerate}
    \item Fidelity and Completeness. Do the visual metaphors present the true
    and complete behavior of the underlying virtual machine?
    \begin{enumerate}
      \item Invasiveness. If the system can be used to visual concurrent
      applications, does its use disrupt the execution sequence of the program?
    \end{enumerate}
    \item Data Gathering Time. Is the data on which the visualization depends
    gathered at compile-time, at run-time, or both?
    \begin{enumerate}
      \item Temporal Control Mapping. What is the mapping between 'program time'
      and 'visualization time'?
      \item Visualization Generation Time. Is the visualization produced as a
      batch job (post-mortem) from data recorded during a previous run, or is it
      produced live as the program executes?
    \end{enumerate}
  \end{enumerate}
  \item Form. What are the characteristics of the output of the system (the
  visualization)?
  \begin{enumerate}
    \item Medium. What is the primary target medium for the visualization
    system?
    \item Presentation Style. What is the general appearance of the
    visualization?
    \begin{enumerate}
      \item Graphical Vocabulary. What graphical elements are used in the
      visualization produced by the system?
      \begin{enumerate}
        \item Colour. To what degree does the system make use of colour in its
        visualizations?
        \item Dimensions. To what degree are extra dimensions used in the
        visualization?
      \end{enumerate}
      \item Animation. If the system gathers run-time data, to what degree does
      the resulting visualization use animation?
      \item Sound. To what degree does the system make use of sound to convey
      information?
    \end{enumerate}
    \item Granularity. To what degree does the system present coarse-granularity
    details?
    \begin{enumerate}
      \item Elision. To what degree does the system provide facilities for
      eliding information?
    \end{enumerate}
    \item Multiple Views. To what degree can the system provide multiple
    synchronized views of different parts of the software being visualized?
    \item Program Synchronization. Can the system generate visualizations of
    multiple programs simultaneously?
  \end{enumerate}
  \item Method. How is the visualization specified?
  \begin{enumerate}
    \item Visualization Specification Style. What style of specification is
    used?
    \begin{enumerate}
      \item Intelligence. If the visualization is automatic, how advanced is the
      visualization software from an AI point of view?
      \item Tailorability. To what degree can the user customize the
      visualization?
      \begin{enumerate}
        \item Customization Language. If the visualization is customizable, how
        can the visualization be specified?
      \end{enumerate}
    \end{enumerate}
    \item Connection Technique. How is the connection made between the
    visualization and the actual software being visualized?
    \begin{enumerate}
      \item Code Ignorance Allowance. If the visualization system is not
      completely automatic, how much knowledge of the program code is required
      for a visualization to be produced for the user?
      \item System-Code Coupling. How tightly is the visualization system
      coupled with the code?
    \end{enumerate}
  \end{enumerate}
  \item Interaction. How does the user of the SV system interact with and
  control it?
  \begin{enumerate}
    \item Style. What method does the user employ to give instructions to the
    system?
    \item Navigation. To what degree does the system support navigation through
    a visualization?
    \begin{enumerate}
      \item Elision Control. Can the user elide information or suppress detail
      from the display?
      \item Temporal Control. To what degree does the system allow the user to
      control the temporal aspects of the execution of the program?
      \begin{enumerate}
        \item Direction. To what degree can the user reverse the temporal
        direction of the visualization?
        \item Speed. To what degree can the user control the speed of execution?
      \end{enumerate}
    \end{enumerate}
    \item Scripting Facilities. Does the system provide facilities for managing
    the recording and playing back of interactions with particular
    visualizations?
  \end{enumerate}
  \item Effectiveness. How well does the system communicate information to the
  user?
  \begin{enumerate}
    \item Purpose. For what purpose is the system suited?
    \item Appropriateness and Clarity. If the automatic (default) visualizations
    are provided, how well do they communicate information about the software?
    \item Empirical Evaluation. To what degree has the system been subjected to
    a good experimental evaluation?
    \item Production Use. Has the system been in production use for a
    significant period of time?
  \end{enumerate}
\end{enumerate}


% (a) scope. what is the range of programs that the SV system may take as input
% for visualization? (a.1) generality. can the system handle a generalized range
% of programs or does it display a fixed set of examples? (a.1.1) hardware. what
% hardware does it run on? (a.1.2) operating system. what os is required to run
% it? (a.1.3) language. what programming language must user programs be written
% in? (a.1.3.1) concurrency. if the programming language is capable of
% concurrency, can the sv system visualize the concurrent aspects? (a.1.4)
% applications. what are the restrictions on the kinds of user programs that can
% be visualized? (a.1.4.1) specialty. what kinds of programs is it particularly
% good at visualizing (as opposed to simply capable of visualizing)? (a.2)
% scalability. to what degree does the system scale up to handle large examples?
% (a.2.1) program. what is the largest program it can handle? (a.2.2) data sets.
% what is the largest input data set it can handle?


% (b) content. what subset of information about the software is visualized by the
% SV system? (b.1) program. to what degree does the system visualize the actual
% implemented program? (b.1.1) code. to what degree does the system visualize the
% instructions in the program source code? (b.1.1.1) control flow. to what degree
% does the system visualize the flow of control in the program source code?
% (b.1.2) data. to what degree does the system visualize the data structures in
% the program source code? (b.1.2.1) data flow. to what degree does the system
% visualize the flow of data in the program source code? (b.2) algorithm. to what
% degree does the system visualize the high-level algorithm behind the software?
  

\section{Previous Work} 

A geometric workbench combines a geometric algorithm library with a geometric
algorithm visualizer. \emph{Geometric algorithm libraries} provide a broad
selection of algorithms and the substrate of types upon which they are built.
\emph{Geometric algorithm visualizers} provide a GUI capable of animating and
visually debugging those algorithms implemented in the library. They borrow
heavily from techniques used in general algorithm animation
systems~\cite{brown1984system, stasko1990tango, stasko1995polka,
stasko1995samba}. Previous works may thus be categorized as
libraries~\cite{mehlhorn1989leda, fabri1998design, overmars1996designing,
fabri1996cgal}, visualizers~\cite{phillips1993geomview, hanson1994interactive,
amenta1995geomview, basken2002geowin}, and full workbench
systems~\cite{schorn1991robust, de1993geolab, de1993animation,
epstein1994workbench, tal1995visualization, shneerson1997gasp,
wei2009geobuilder}. Our review focuses on the latter two.

\subsection{Geometric workbench systems}

\paragraph{XYZ GeoBench (1991)}

XYZ (eXperimental geometrY Zurich) GeoBench was a geometric workbench developed
by Peter Schorn under the supervision of Jurg
Nievergelt~\cite{schorn1991robust}. Schorn's 1991 thesis focused on the question
of how to produce good software for geometric computation, with a particular
emphasis on geometric robustness. Schorn describes the GeoBench as ``a
programming environment, implemented in an object oriented language, for the
rapid prototyping of geometric software and a testbed for experiments,'' noting
that ``algorithm animation is used for demonstration purposes and debugging.''
The XYZ Library was built on top of the GeoBench and offered implementations of
a large number of geometric algorithms.

There are several notable features about the GeoBench. The first is the use of
interchangable arithmetic and software simulation of parameterized floating
point numbers. In particular, algorithm implementations make reference to an
abstract 2D point class that does not specify a particular number system for the
coordinates. The abstract point class is extended to form concrete point types
for single-precision (realPoint), long integer (longIntPoint), and parameterized
floating point (floatPoint) coordinate types. Counterinuitively, the
parameterized floating point arithmetic was not used to ameliorate robustness
problems by increasing precision, rather it was used to simulate low precision
floating point to make the robustness problems more pronounced.

% The GeoBench had three major goals or desiderata. First was to be a programming
% environment. Provide the implementor of geometric algorithms with the necessary
% infrastructure for rapid prototyping. Ingredients of this environment include
% rich set of geometric primitives, fundamental set of basic geometric algorithms,
% collection of abstract data types and the ability to perform ``universal
% operations'' like input/output or scaling on geometric objects. Second was to be
% an interactive testbed for experiments. Wanted to measure an implementation's
% efficiency by comparing it to other programs solving the same problem. Would
% like to experiment with different implementations of abstract data structures
% and to try different models of arithmetic. Need to construct degenerate
% configurations as test cases and save them in a test suite. Third was algorithm
% animation. Most algos can be animated fairly easily since geometric objects have
% clear standard graphical representations. algo animation is used for
% demonstration in the classroom and for debugging.

\paragraph{Workbench (1991)}

Workbench is a system similar to XYZ GeoBench that focused on implementing
complex geometric algorithms instead of robustness
issues~\cite{epstein1994workbench}. The system is built in Smalltalk and
composed of three main components: a library, visualization GUI, and tools for
extending each. It focused heavily on empirical comparisons of different
algorithms and using animations for teaching and demonstration. They define the
minimum criteria for a geometric workbench as a system with: representations of
geometric objects, geometric data types, non-geometric data types, algorithmic
implementations that adhere to specification, different arithmetic types, and a
GUI that provides animation and debugging facilities.

% they identify two other projects (LEDA and GeoBench): LEDA had a large number of
% graph algorithms and well designed data types but was just then starting to move
% toward geometry, GeoBench is similar to workbench (UI + library) but main focus
% is on robust implementation of fundamental algorithms. the algorithmic portion
% of their software is layered with primitive operations and types on the bottom
% and more complex types on top. the ui component is separate and similarly
% layered. the primary goal of this project was to create a geometric computing
% environment: a tool for geometric computation applicable in a variety of
% contexts that provides useful facilities for dealing with the complex algorithms
% typical of computational geometry. they identify three main components such an
% environment must have: library of algs and data types, GUI for manipulating
% library objects, and tools for enhancing and extending the lib and GUI.
% empirical comparison of different algorithms and data structures is important,
% display and animation features are useful for teaching/demonstration. they
% consider the following worthwhile but didn't pursue them: optimizing
% implementations, handling degeneracies and numerical problems.
% 
% they define the minimum criteria for a geometry workbench as having:
% representations of geom objects (polygons etc), geometric data types/structures,
% nongeometric data types/structures (splay trees, heaps), algorithmic
% implementations adhering to specification, different arithmetic types, GUI with
% algorithm animation, programming environment and debugging facility (I/O of
% geometric objects). they implement the workbench in smalltalk

\paragraph{GeoLab (1993)}

de Rezende created GeoLab, a system that binds together support for software and
algorithm development with realtime interaction~\cite{de1993geolab,
de1993animation}. The system supports software development with built-in
abstract data types for geometric objects, data structures, basic algorithms,
and mechanisms for incorporating new components \emph{without} recompilation.
The system supports user interaction with the ability to construct input data,
debug implementations visually, gather statistics, input/output geometric data,
and customize algorithm animations. Although the system is implemented in
object-oriented C++, algorithms are not necessarily member functions of the
classes upon which they operate -- they may be free functions. This echoes
Meyer's advice that minimal use of member functions can often increase
encapsulation.

% software
% development:
% build in ADT's for geometric types, data structures, basic algorithms and some
% complex geometric data structures, mechanisms for incorporation of new
% components WITHOUT recompilation of the environment, etc.
% support for interaction: GUI, ability to construct input data, debugging and
% statistics, I/O, customization of algorithm animation. written in C++.

% new modules are imported into the system via shared libraries, the kernel itself
% contains no geometric code. GUI consists of editing area, operation pallete,
% algorithms menu. random generators for input data. double hierarchy of classes -
% pure objects and graphics objects. for each pure geometric object (e.g. point2d)
% there is a graphical geometric counterpart. geometric algorithms are not
% necessarily methods of the classes they operate on - they may be free functions.

\paragraph{GASP (1995)}

Tal and Dobkin describe the Geometric Animation System, Princeton
(GASP)~\cite{tal1995visualization}. The system allows users to quickly create 3D
visualizations, animate complex geometric algorithms, and visually debug
implementations. The use of style files to control the visual aspects of the
animation allows users to produce new renderings without recompiling the system.

% It can be used as an illustration tool for geometric
% constructions, to create videotapes to accompany talks, as a debugger, animator,
% and as an instructional aid to students. 

% three objectives set them apart from other
% animation systems such as Balsa Balsa-II Tango and Zeus: quick creation of 3d
% visulizations, can animate complex geometric algorithms, includes a visual
% debugging facility. the system may be used in many ways: illustration tool for
% constructions, videotapes to accompany talks/classes, for debugging, allow
% students to interact and experiment with animations, allows users to create
% animations to attach to their documents. previous systems identified two user
% types (client programmer and end user), they define three: end user, naive
% programmer (animations are easy), advanced programmer (can change the animation
% around). uses style files to control visual aspects of animations. thus, there
% are 4 parts to an animation: animation system, algorithm implementation, hooks
% to animation system inside implementation, style files. programmers concentrate
% on logical operations that need to be visualized (the what) but not how to do it
% (the how). identify four-dimensional space as future work.

%\paragraph{GASP-II (1998)}

%todo~\cite{shneerson1997gasp}.
% GASP-II had something to do with an electronic
% classroom~\cite{shneerson1997gasp}.

%\paragraph{WAVE (1999, 2001)}

%todo~\cite{baker1999visualizing, demetrescu2001visualizing}.

% tamassia and liotta got into visualizing geometric algorithms over the web in
% 1999 and 2001~\cite{baker1999visualizing, demetrescu2001visualizing}.

\paragraph{GeoBuilder (2009)}

GeoBuilder is a cross-platform geometric workbench written in
Java~\cite{wei2009geobuilder}. It features a novel mechanism for automatically
positioning the 3D camera during algorithm execution and allows users to
collaborate on programming and visualization. Wei et al demonstrate the 3D
tracking feature by constructing convex hulls and detecting line segment
intersections. They note that automatically positioning a single view of an
algorithm may not be sufficient to capture all changes in algorithm state, and
identify automatic positioning of multiple cameras for future work.

\subsection{Geometric algorithm libraries and visualizers}

\paragraph{LEDA}

The library of efficient data structures and algorithms (LEDA) is often
mentioned in the context of workbenches.

\paragraph{CGAL}

The computational geometry algorithms library (CGAL) is the current standard
implementation among the geometry community - it used to provide support for
GeomView but now just has its own visualization module.

\paragraph{Geomview}

Geomview is a geometric visualization system that focuses on rendering and
manipulating geometric data in 3-space~\cite{phillips1993geomview,
hanson1994interactive, amenta1995geomview}. It is able to render both static
geometry and dynamic geometry produced by an external program running
concurrently. External programs that use Geomview in this fashion are called
``external modules.'' It gained a large userbase after its initial release in 1991
due to its decoupled approach to animating geometric algorithms; users could
implement algorithms however they wanted and simply interface with Geomview at
runtime. Of particular note is its 4D visualization module that allow users to
explore 4-dimensional geometry through various projections. CGAL provides a
module for producing Geomview visualizations.

\paragraph{GeoWin}

GeoWin is a C++ data type that provides the ability to visualize sets of
geometric objects, with a focus on two dimensions~\cite{basken2002geowin}. This
functionality may be used to visualize the output of geometric algorithms or to
animate the progression of algorithms. GeoWin defines a programming interface to
define scenes and an interactive interface to define how the user may interact
with the scene. The data type integrates directly with LEDA and CGAL while
custom libraries must implement the interfaces, entailing a dependency on LEDA
types.


\FloatBarrier
\section{DDAD Workbench Desiderata}

Geometric objects on a digital computer are composed of two types of data:
numerical and combinatorial. Examples of numerical data include the
Cartesian coordinates of a point in 3-space, the length of a line segment
connecting two such points, or the angle between two such line segments.
Examples of combinatorial information include grouping two points as an
edge, grouping a collection of edges as a face, or grouping a collection of
faces as a surface.

Geometric algorithms that operate on geometric objects are best thought of as
two types of operations: predicates and constructions. Predicates determine
relationships between objects. A predicate might determine if a point is to
the left, right, or is collinear with a line segment, determine if a point is
inside, outside, or on a circle, or determine if a line intersects a plane in
one, none, or infinitely many points. Constructions produce new geometric
objects from existing geometric objects. A construction might produce the
rotation of a point around an origin, produce the point of intersection
between two line segments, or produce an offset of an algebraic curve.

\paragraph{Scope} describes the range of programs the system can take as input
for visualization. First, the system should visualize geometric algorithms
implemented with primitives and predicates from the degree-driven algorithm
design library (DDAD). Second, the system should provide example implementations
and visualizations of algorithms that solve textbook computational geometry
problems. Third, the system should be flexible enough to visualize new geometric
algorithms from the facilities used in example visualizations. Fourth, the
system should run well given inputs large enough to verify the correctness of a
particular implementation.

\paragraph{Content} describes what subset of information about the software is
visualized by the SV system. For algorithm implementers, the system should
visualize information about the concrete implementation of the algorithm. This
entails combining traditional debugging facilities with the new ability to
visualize geometric structures and operations. For algorithm presenters, the
system should visualize information about the high-level algorithm description.
This entails mechanisms for displaying visualizations that do not necessarily
correspond to the underlying implementation details.

\paragraph{Form} describes the characteristics of the output of the system (the
visualization). First, the output medium should be an interactive computer
program. This is sufficient for implementers, but presenters may want to produce
a digital recording of an algorithm animation. Fortunately, screen capture
programs can facilitate producing this type of output. Second, the output should
have a rich graphical vocabulary, displaying standard graphics primitives
(points, lines, polygons), each with a variety of attributes (color,
transparency, pattern, texture). Presenters in particular desire highly
expressive output. Third, the output should contain both 2D and 3D elements.
The third dimension can be used both for visualizing inherently 3D algorithms
and to effectively display non-dimensional information. Fourth, the output
should have multiple levels of information granularity, with the ability to
elide and reveal specific levels as needed. Fifth, users should be able to view
the same data in multiple ways. This capability is particularly pertinent in
geometry where insight into a problem may be gained from viewing a dual
formulation (\emph{e.g.} viewing points as lines).

\paragraph{Method} describes how the visualization is specified. Implementers
desire highly automated specifications that minimally invade implementation source
code. Increased automation increases the likelihood a tool will be used for
debugging assistance. Useful automated visualizations require a high level of
intelligence in order to correctly recognize and display high level data
structures. In contrast, presenters are less concerned with automated
visualization specifications, and accept a higher degree of invasiveness in
return for customization and increased expressibility. The system should strike
a balance between automation and customization.

\paragraph{Interaction} describes how the user of the software visualization
system interacts with and controls it. First, users should be able to control
the speed of execution. This includes starting, stopping, speeding up, slowing
down, and single stepping. Second, users should be able to control a camera to
navigate around the scene and to focus on different objects. This is especially
important for 3D algorithms and larger data sets. Third, users should be able to
record and play back particular runs of a visualization. For presenters this
could aid pedagogy, and implementers might record when an algorithm fails.

\subsection{Current Capabilities}

The workbench currently satisfies the desired functionality as follows. It
extends the DDAD library with visual types and encapsulates these into a
geometry kernel. The geometry kernel is paired with a visualizer that is capable
of animating algorithm execution. The workbench strikes a balance between
automated visualizations and expressive power by using object
oriented design to encapsulate visual commands in lower level objects. The user
is able to control the speed of execution and pause at important moments. They
control two different views of the scene and can move these views interactively
as the algorithm executes.


%==============================================================================
% @author Clinton Freeman <freeman@cs.unc.edu>
% @date 2014-05-23
%==============================================================================

\FloatBarrier
\section{Case Study: Melkman's Algorithm}
\label{sec:case-melkman}

In this section, we use the workbench to implement and visualize Melkman's
convex hull algorithm~\cite{melkman1987line}. The algorithm takes as input a
simple polyline and produces as output a convex polygon (represented as a
deque). Our workbench already has these types implemented, complete with
visualization code for each basic operation. After explaining how the algorithm
works, we introduce the \texttt{Polyline\_2r} and \texttt{Polygon\_2r} types and
use them to arrive at a correct implementation. We conclude by using the GUI to
generate an input polyline and run the algorithm.

%, and save the output polygon.

% First, we implement the basic data
% types used by the algorithm and augment their methods with visualization
% code. Second, we use these data types to implement Melkman's algorithm. 
% By visualizing the data types in an object-oriented way, we arrive at a
% clean implementation of the algorithm.

% This section examines using our workbench to animate Melkman's convex hull
% algorithm~\cite{melkman1987line}. 

%==============================================================================

\subsection{Algorithm Overview}

A \emph{polyline} $P$ is a polygonal chain of vertices $p_1, p_2, \ldots, p_n$
connected by line segments $\seg{p_ip_{i+1}}$ for $1 \leq i < n$. $P$ is
\emph{simple} if the only intersection between segments is at their shared endpoints.
Melkman's algorithm incrementally computes the convex hull of a simple polyline
in $O(n)$ time. 

The algorithm stores the hull's vertices in a doubly-ended queue (deque) and
maintains the invariant that they are stored in ccw order from head to tail,
starting and ending with the most recent vertex added to the hull. The algorithm
establishes the invariant initially by forming the deque with $p_2, p_1, p_2$ to
represent the convex hull of the first two points. 

Now, suppose we wish to add $p_i$ to the hull. Let $v, w$ be the vertices at the
tail of the deque and $u, v$ be the vertices at the head. Thus, $v$ is the most
recent vertex added to the hull, and we can speak of edges $\seg{uv}$ and
$\seg{vw}$ as being at the head and tail of the deque, respectively. 

If $p_i$ is not left of $\seg{uv}$ or inside $\seg{uv}$, then remove edge
$\seg{uv}$ from the convex hull by popping the head of the deque; continue until
$p_i$ is left of the edge at the head. Similarly, if $p_i$ is not left of
$\seg{vw}$ or inside $\seg{vw}$, then remove edge $\seg{vw}$ from the convex
hull by popping the tail of the deque; continue until $p_i$ is left of the edge
at the tail. Finally, push $p_i$ onto both the head and tail of the deque to
restore the invariant.

On the other hand, if $p_i$ is left or inside both $\seg{uv}$ and $\seg{vw}$,
then we can observe that $p_i$ is not on the convex hull: because the polyline
from $v$ to $p_i$ does not cross the polyline from $u$ to $v$ or from $v$ to
$w$, $p_i$ can leave the hull $CH(P_{i-1})$ only by crossing $\seg{uv}$ or
$\seg{vw}$. Hull $CH(P_{i-1})$ is identical with $CH(P_i)$, and the invariant
already holds.

%==============================================================================

%\subsection{Building Blocks Provided by the Workbench}

\subsection{Algorithm Implementation}

Melkman's algorithm involves two geometric objects (polylines and polygons), and
requires only a single degree-two predicate to check whether a point is to the
left or inside of a directed line segment. The DDAD workbench provides both
geometric types (\texttt{Polyline\_2r} and \texttt{Polygon\_2r}), and the
predicate \texttt{RIsLeftOfOrInsidePQ}. Both geometric types are built using a
deque as an underlying data store, so we will have the push and pop
capabilities we need. All two dimensional geometric types are given methods to
set the $z$ coordinate (or \emph{z-order}) of all its vertices so that users can
draw some objects above or below others. 

The final type we must use is \texttt{IGeometryObserver}. All DDAD geometric
types are built to be observed by other types that implement the
\texttt{IGeometryObserver} interface. Geometric algorithm implementations that
wish to be visualized will usually be a free function with an
\texttt{IGeometryObserver} as the last input argument. The only thing we need to
do with the \texttt{observer} is to have it observe the output polygon before we
perform any operations it.

Code listing~\ref{melkman-function} shows \texttt{Melkman}, the final 
implementation of Melkman's algorithm. The function takes as input a 
\texttt{Polyline\_2r} and an \texttt{IGeometryObserver}, and produces as output
a \texttt{Polygon\_2r} object. All branching tests make use of the
\texttt{RIsLeftOrInsidePQ} predicate. Aside from a few lines setting up colors,
z-order, and observing the output hull, the implementation is a straightforward
transcription of the algorithm, uncluttered by visualization code. 
 
\lstinputlisting[float,caption=Melkman
implementation,label=melkman-function]{code-samples/melkman.cpp}

% \lstinputlisting[float,caption=Polyline\_2r
% class declaration,label=polyline-class]{code-samples/polyline.cpp}

% The real work of implementing Melkman's algorithm is designing the data types
% upon which it operates. In particular, we require data types for two geometric
% objects: input polylines and output polygons. Beyond the standard concerns of
% efficiency and ease of use, both data types must support visualization and our
% polygon type must support deque semantics.
% 
% Our workbench provides an abstract data type, \texttt{Visual::Geometry}, that
% geometric data types can inherit to become \emph{visual geometry} types and gain
% access to the visualization system. We provide an in-depth discussion of how
% \texttt{Visual::Geometry} works in Section~\ref{sec:workbench-architecture}, but
% for now it is sufficient to know that uses the observer
% pattern~\cite{gamma1994design} to generate and recieve visual events. In
% particular, visual geometry objects are able to both observe visual geometry
% objects and to be observed by visual geometry objects. A visual geometry object
% may \emph{signal} to its observers that a visual event has occurred, and can
% optionally implement a \emph{slot} that handles signals from the objects that it
% observes. By default, visual geometry objects simply forward signals they
% recieve up to their observers, forming an event propagation chain.
% 
% 
% A major benefit of visual geometry types is their ability to offload
% visualization work through composition. Our polygon type must support deque
% operations, but we only require sequential access to the polyline's vertices. In
% our case, we can think of the polygon as being composed of a polyline boundary.
% This suggests using a deque as a backing store for the polyline's vertices so
% that the polygon type may simply delegate deque operations to its boundary. This
% choice is consistent with our need for constant sequential access to the
% polyline's vertices.



% The \texttt{Polygon\_2r} class is composed of a \text{Polyline\_2r} boundary
% that is responsible for visualizing vertices and edges. It can optionally
% visualize the polygon interior by triangulating it into a fan.  

%  We know from the algorithm description that these should support deque
% semantics, so it makes sense to use this as a backing store for the polygon's
% vertices. In many regards, a polygon will act like a polyline, e.g. we can add
% and remove vertices to both. In order to avoid duplicating functionality, it
% makes sense to compose our polygon with a polyline boundary and forward common
% events to it. We want to visualize both objects, so each will inherit from the
% Visual::Geometry class.

% The \texttt{Polyline\_2r} class uses a deque to store its vertices. The
% \texttt{push\_back} and \texttt{pop\_back} methods are responsible for
% visualizing how those operations affect the visual state of the object.



% \lstinputlisting[float,caption=Polygon\_2r
% class declaration,label=polygon-class]{code-samples/polygon.cpp}

% \lstinputlisting[float,caption=Polyline\_2r
% push back
% implementation,label=polyline-push-back]{code-samples/polyline-push-back.cpp}
  
%==============================================================================

%==============================================================================

\subsection{Generating Input Data and Executing the Implementation} 

The workbench GUI is composed of a toolbar and two views: orthographic and
perspective. The toolbar contains input buttons and a button for turning
snapping on and off. The orthographic view contains an integer grid and allows
the user to zoom and pan the camera. The perspective view renders the scene in
3D and allows the user to move and rotate the camera.

The orthographic view provides a simple CAD interface, and the input buttons
control which objects are created when the user clicks inside the view. One
button allows the user to create polylines. The snapping toggle button
determines whether the points created are integral. 

After clicking a few times to create a polyline, the user completes the object
by right-clicking, and the polyline remains selected. Right-clicking again
reveals a menu with various algorithms listed. With a few lines of code, we can 
add Melkman to the list and specify that our new function should execute on
the currently selected object when we click on it. 


 
%==============================================================================

% \begin{figure}[h]
% 	\centering
% 	\includegraphics[width=\textwidth]{figures/melkman-input-1}
% 	\caption{Example polyline input.}
% 	\label{fig:melkman-input}
% \end{figure}

 
%As in Figure 1.18,

% Figure 1.19 continues the execution begun in Figure 1.16. It shows all of the
% deques and some of the hulls for $CH(P_5)$ through $CH(P_{14})$. Code is listed
% in Figure 1.20.

% It can also be used to compute the convex
% hull of arbitrary point sets if we first sort by $x$ coordinate, breaking ties
% by $y$ coordinate.

% Melkman's algorithm stores the convex hull vertices in a deque, or doubly-ended
% queue - a simple data structure that stores a list of elements and allows you to
% add and remove (by push() and pop()) elements from the front and back of the
% list. Melkman's algorithm maintains the invariant that the vertices of the
% convex hull CH(P\_i) are stored in a deque in ccw order from head to tail,
% starting and endign with the most recent vertex added to the hull.
% 
% Given an oriented line $pq$ and a point $r$, the 2D orientation predicate
% $\textsc{Orient2D}(p, q, r)$ answers the question, ``is $r$ to the left, right,
% or on $pq$?'' It is often written as the sign of the 2-by-2 determinant, $$
% \textsc{Orient2D}(p, q, r) = \textsc{Sign}\left( \begin{vmatrix} p_x-r_x &
% p_y-r_y \\ q_x-r_x & q_y-r_y \end{vmatrix} \right).$$

% Incremental convex hull algorithms construct the hull by examining each input
% point in turn, exploiting structure in the partial hull to help reduce
% computation. 
% Graham and Yao observed that if we know the points in advance, we
% may make our task easier by considering them in sorted order by $x$ coordinate,
% breaking ties by $y$ coordinate. Then point $p_i$ will always be a vertex of the
% convex hull $\text{CH}(P_i)$, and it will either be adjacent to $p_{i-1}$ or
% will cause $p_{i-1}$ to be removed from $\text{CH}(P_i)$.
 
% \begin{mdframed}[linecolor=white, backgroundcolor=algback, frametitle={Algorithm
% Melkman}] 
% \begin{algorithmic}[1]    
%     \Require Simple polyline $P = \langle v_1, \ldots, v_m \rangle$.
%     \Ensure $\text{CH}(P)$.
%     \vspace{0.75em}
%     \Procedure{Melkman}{$P$}
%     \State $H.push\_back(v_2); H.push\_back(v_1); H.push\_back(v_2);$
%     \Comment{Init hull}
%     \For{$i=3\ldots m$}
%     	\If{\textsc{!LeftOrInside}$(H.back(1), H.back(0), v_i)$ or
%     	\textsc{!LeftOrInside}$(v_i, H.front(1), H.front(0))$} 
%     		\While{\textsc{!LeftOrInside}$(H.back(1), H.back(0), v_i)$}
%     			\State $H.pop\_back();$
%     		\EndWhile
%     		\While{\textsc{!LeftOrInside}$(v_i, H.front(1), H.front(0))$}
%     			\State $H.pop\_front();$
%     		\EndWhile
%     		\State $H.push\_back(v_i);$
%     		\State $H.push\_front(v_i);$
%     	\EndIf
%     \EndFor
%     \State \Return $H$
%     \EndProcedure
% \end{algorithmic}
% \end{mdframed} 

% Animating an algorithm using the workbench is composed of a number of tasks,
% namely 
% \begin{itemize}
%   \item Implement input data structures and instrument them with visualization
%   code.
%   \item Optionally modify the GUI to allow the user to create instances of
%   the input data structure.
%   \item Implement output data structures and instrument them with visualization
%   code.
%   \item Implement predicates.
%   \item Implement the algorithm and instrument it with a small amount of
%   visualization code.
%   \item Optionally modify the GUI to allow the user to run the algorithm on
%   selected input data. 
% \end{itemize}

% \subsection{Data Structures}
% 
% polychain\_2r
% 
% polygon\_2r

% Animating any algorithm begins with generating the appropriate input. In the
% case of Melkman's algorithm, we begin by creating a simple polyline. This is
% accomplished by the user clicking on points in the 2D top-down orthographic
% view. The user may choose to place integer vertex coordinates or turn off
% snapping so that vertex coords are floating point or rational coords.
% 
% 
% 
% \begin{lstlisting}
% Polygon_2r Melkman(const PolyChain_2r& P, Visual::IGeometryObserver* ge_obs) {
%     Polygon_2r CH_P;
% 
% 
%     return CH_P;
% }
% \end{lstlisting}


\FloatBarrier
\section{Workbench Architecture}
\label{sec:workbench-architecture}

\begin{figure}[htb]
\centering
\includegraphics[width=\textwidth]{figures/components-uml-4} 
\caption{An abstract overview of the workbench architecture.}
\label{fig:components} 
\end{figure}

The workbench comprises two major components: a geometric algorithm library and
a geometric algorithm visualizer. The \emph{geometry library} implements
geometric algorithms. It defines types for arithmetic, linear algebra, and
geometric primitives (e.g. points, line segments, and triangles). These are
building blocks with which it defines more complex geometric objects (e.g.
polygons and polytopes). All of these may be used to implement geometric
algorithms. The \emph{visualizer} displays the execution of geometric
algorithms. It is structured using the model-view-controller (MVC) design
pattern. In particular, it maintains a visual model, several views of the visual
model, and a user interface that functions as the controller. In this section,
we are concerned with explaining how the kernel and visualizer work together to
produce algorithm visualizations.

\FloatBarrier
\subsection{The Geometry Library}

The \emph{arithmetic} and \emph{linear algebra} components are orthogonal to our
discussion of visualization. The \emph{graphical vocabulary} component is
necessary to define briefly but is straightforward to understand and depends
only on fundamental language types. In particular, it consists of types to
specify color, transparency, and a lighting model.

\emph{Geometric primitives} are geometric objects of constant size. We are
concerned with three: points, line segments, and triangles\footnote{Other
primitives include lines and rays but these are not directly supported by the
visual interface.}. Points are the basic unit of visualization. They are
equipped with unique identifiers (UIDs), assigned by registering with the visual
model. Line segments and triangles are combinatorial groupings of two and three
points, respectively. Thus, they are implicitly equipped with unique identifiers
given by the combinatorial grouping of their points' UIDs.

\emph{Visual primitives} are types that define the subset of graphical
vocabulary applicable to a corresponding geometric primitive. For
example, a visual point type might specify that points may be assigned a color
and rendered as either a circular or square sprite; a visual segment type might
also specify a color attribute but have no notion of a sprite shape.

The \emph{visual event system} defines the notion of observable geometry.
\emph{Observable geometry} objects notify their observers of changes in their
visual state by emitting signals. \emph{Geometry observers} handle these
notifications by implementing a corresponding slot. All higher level geometric
types that wish to be rendered by the visualizer must implement the observable
geometry interface. If a geometric object is composed of other observable
geometry objects, then it must also implement the geometry observer interface
and subscribe to those other objects.

Many geometric types will be composed of other observable geometry objects, so
we define an abstract base class that is both observable and capable of
observing other objects. By default, this base class forwards any visual events
it receives from the objects it observes onward to its own subscribers. In this
way, visual events are passed through a chain of event handlers, eventually
arriving at the scene observer in the visualizer.

\begin{figure}[htb]
\centering
\includegraphics[width=\textwidth]{figures/visual-uml-6} 
\caption{The visual event system embedded in the geometry kernel.}
\label{fig:visual} 
\end{figure} 

\FloatBarrier
\subsection{The Visualizer}

\emph{Scene objects} are wrappers around geometric objects that implement an
interface required by the scene observer (e.g. being selectable or having a
name). Scene objects observe the geometric object they wrap, and may be observed
by the scene observer.

The \emph{scene observer} is the ultimate destination for all visual events
produced by geometric objects. It maintains three maps that pair unique
geometric primitives with a stack of visual primitives, the top of which defines
the current visual state. The scene observer is responsible for translating the
graphical vocabulary defined in the kernel into OpenGL buffer objects. The scene
observer lives inside of its own thread, and each of the visual events emitted
from an observable object may be given an integer value that will pause this
thread for a corresponding number of milliseconds.

The \emph{renderer} is responsible for managing OpenGL data and state.The
\emph{orthographic view} provides a top-down orthographic projection of the
scene. The user may pan around the plane and zoom in/zoom out. The
\emph{perspective view} provides a perspective projection of the scene. The user
may arbitrarily orient the camera and travel forward and backward along the view
direction.


%==============================================================================
% @author Clinton Freeman <freeman@cs.unc.edu>
% @date 2014-11-26
%==============================================================================

\FloatBarrier
\section{Case Study: Randomized Incremental Delaunay Triangulation}
\label{sec:case-delaunay}

In this section, we use the workbench to implement and visualize a randomized
incremental Delaunay triangulation algorithm~\cite{lischinski1994incremental}.
We use the algorithm to construct \emph{polyhedral terrains}, or graphs of
continuous functions that are piecewise linear. The algorithm takes as input a
special point set called a terrain. A \emph{terrain} is a 2-dimensional surface
in 3-dimensional space with a special property: every vertical line intersects
it in a point, if it intersects it at all. The algorithm produces as output a
triangulation that is \emph{Delaunay}: the circumcircle of any triangle in the
triangulation does not contain any input points in its interior. 

\begin{figure}[htb]
\centering
\shadowimage[width=\textwidth]{figures/delaunay-terrain-intro} 
\caption{A polyhedral terrain produced by the workbench.}
\label{fig:terrain-intro} 
\end{figure}

Our workbench already has a simple point set type
(\texttt{PointSet\_3r}), but we will need to introduce a new type 
(\texttt{Terrain\_3r}) and visualize its methods in order to animate the
algorithm. The algorithm uses the QuadEdge data
structure~\cite{guibas1985primitives} to store and manipulate terrain topology.
After explaining how the algorithm and data structure work, we use Heckbert's
QuadEdge C++ library to implement \texttt{Terrain\_3r} and use it to arrive at a
correct implementation of incremental Delaunay. We conclude by using the GUI to
generate an input point set and run the algorithm.

%==============================================================================

\subsection{Algorithm Overview}

The algorithm starts with an initial triangle, constructed such that it contains
all input points and is large enough to not affect the points' triangulation.
Points are inserted one at a time in a random order. Each insertion maintains
the invariant that the triangulation is Delaunay. There are three steps to
inserting a point $p$: localizing $p$ to a triangle $T$, inserting $p$ into
$T$, and restoring the Delaunay invariant. The insertion is a constant time
operation that adds three new edges from the triangles vertices to $p$. Below,
we describe localizing $p$ and restoring the invariant.

We start by finding the triangle that contains $p$. We could achieve optimal
$O(\log n)$ time, but this requires maintaining a complicated data structure.
Instead, we use the naive strategy of iterating over all triangles,
performing containment tests.  This strategy is incredibly easy to implement and 
- if the points are randomly chosen from a uniform distribution - only requires
$O(n^{1/2})$ operations in expectation. Once we locate $T$, we insert $p$ and
move to restore the Delaunay invariant.

We restore the invariant through a series of edge flips. Let $T = abc$ be the
old Delaunay triangle containing $p$, with circumcircle $C$. Then the new edges
$\seg{pa}$, $\seg{pb}$, and $\seg{pc}$ are Delaunay: the circle passing through
$p$ and tangent to $C$ at $a$ is a site-free witness for the Delaunayhood of the
edge $\seg{pa}$ (we can argue analogously for $\seg{pb}$ and $\seg{pc}$). The
edges $\seg{ab}$, $\seg{ac}$, $\seg{bc}$ are {\em suspect} since we do not know
if they pass the \textsc{InCircle} test with respect to $p$ and the triangle on
their other side. We have to check these suspect edges; if an edge fails the
\textsc{InCircle} test then it will be swapped, creating a new Delaunay edge
emanating from $p$ and creating two new suspect edges that must now be tested.
This process continues recursively until the invariant is restored.

%==============================================================================

\FloatBarrier
\subsection{The QuadEdge Data Structure}

\lstinputlisting[float, caption=Visualizing MakeVertexEdge,
label=delaunay-makevertexedge]{code-samples/delaunay-makevertexedge.cpp}

\lstinputlisting[float, caption=Visualizing MakeFaceEdge,
label=delaunay-makefaceedge]{code-samples/delaunay-makefaceedge.cpp}

\FloatBarrier
\subsection{Algorithm Implementation}
 
The incremental DT algorithm involves one geometric object (a subdivision), and
requires a single degree-four predicate to check whether a point is in, on, or
outside the circle defined by three points. The DDAD workbench provides the
geometric type (\texttt{Subdivision\_2r}) and the predicate
(\texttt{InCircle\_2r}).  

\lstinputlisting[float, caption=Incremental Delaunay Implementation,
label=delaunay-function]{code-samples/delaunay-terrain.cpp}

\lstinputlisting[float, caption=Adding a Sample,
label=delaunay-add-sample]{code-samples/delaunay-add-sample.cpp}

\lstinputlisting[float, caption=Locating a Point,
label=delaunay-locate-point]{code-samples/delaunay-locate-point.cpp}

\lstinputlisting[float, caption=Flipping Edges,
label=delaunay-test-and-swap-edges]{code-samples/delaunay-test-and-swap-edges.cpp}

%==============================================================================

\subsection{Generating Input Data and Executing Incremental Delaunay} 


\FloatBarrier
\section{Lessons Learned and Future Directions} 

Implementing a geometric algorithm workbench is a challenging task with a rich
set of problems encompassing a variety of disciplines. We converged on an
interesting events design in which the user annotates observable objects to
signal changes in their visual state. The user specifies visual state on points,
line segments, and triangles, determines delay length between animation
sequences, and directs the viewing camera while the animation runs.
	
Many opportunities for improvement and further exploration exist. First,
implementing existing plans of debugging controls affords an immediate
improvement in the tool's usefulness. Second, allowing the presenter to specify
sounds to accompany interesting events provides another means of conveying
information. Finally, creating better abstractions of common data structures and
investigating integration with existing debuggers reduces system invasiveness
for the implementer.

\bibliographystyle{abbrv}
\bibliography{../../references}

\end{document}

\endinput












































% The workbench is composed of two major systems: the geometry kernel and the
% visualizer. The geometry kernel houses arithmetic (MPIR bignums, number
% theoretic functions), linear algebra (vectors and matrices), low-level geometric
% primitives (points, line segments, and triangles), high-level geometric types
% (polylines, polygons, subdivisions, etc.), and geometric algorithms (convex
% hull, integer hull, delaunay triangulation, etc.). 
% 
% 
% The geometry kernel encapsulates geometric algorithm implementations.
% 
% The visualizer encapsulates the visual display of geometric algorithms.


% Geometric algorithms are typically designed and analyzed using the Real-RAM
% model of computation \cite{preparata1977convex}. In other words, these
% algorithms assume that the numerical data in geometric objects are exact
% values in $\mathbb{R}$ that can be stored and retrieved in constant time, and
% that arithmetic involving these values is performed in constant time. From a
% practical point of view, it may seem like an odd or frustrating decision to
% assume access to infinite precision real arithmetic, given that digital
% computers are finite objects. From a theoretical point of view, this is a
% sensible choice given that for subsets of $\mathbb{R}$, such as the rational
% numbers $\mathbb{Q}$, many fundamental geometric axioms no longer hold.


% \subsection{Report Overview}
% 
% Lacking a suitable workbench, we decided to build our own. This report details
% the resulting system and is composed of two major sections. First, we state
% workbench desiderata, explain current workbench capabilities, and show how these
% capabilities support the desiderata. Second, we overview our workbench
% architecture. We discuss at a high level how the two major components -- the
% geometry kernel and the visualizer -- work together to animate geometric
% algorithms. Finally, we conclude by reviewing lessons learned from the project
% and identify avenues for future work.

% \subsubsection{Forrest geometric computing environments}
% 
% Forrest suggested the geometry community start building ``geometric computing
% environments'' in the late 1980's~\cite{forrest1987computational,
% forrest1988geometric}.

% \subsubsection{LINETool (1988)}
% 
% Yap and Ericson discuss the design of LINETool, a geometric editor. They
% differentiate their tool from automated theorem provers for geometry. Designed
% in support of exact computation.~\cite{ericson1988design}

% There is a long history of geometric workbench systems. In the late 1980's
% Forrest identified the need for ``geometric computing environments'' that would
% provide both a library of geometric algorithms and a supporting user interface
% for animation and debugging~\cite{forrest1987computational,
% forrest1988geometric}. 
% 
% 
% 
% 
% Concurrently, Yap and Ericson designed LINEtool, a
% ``geometric editor" that allowed users to exactly specify and render geometric
% constructions~\cite{ericson1988design}. Soon after, three projects were underway
% that sought to provide a geometric computing environment: XYZ
% GeoBench~\cite{schorn1991robust}, Workbench~\cite{epstein1994workbench}, and the
% Library of Efficient Data Types and Algorithms (LEDA)~\cite{mehlhorn1989leda}.
% XYZ GeoBench and Workbench most fully fit the definition of a geometric
% computing environment, while LEDA focused primarily on the development of a
% library. 
% 
% Dobkin and Hausner~\cite{hausner1999animation} review four geometric
% visualization systems: Workbench \cite{epstein1994workbench}, XYZ
% GeoBench \cite{schorn1991implementing, schorn1991robust}, Geomview
% \cite{amenta1995geomview}, and GASP \cite{tal1995visualization}.

% Schorn joined
% Nievergelt's group at UNC Chapel Hill in 1986 and completed his MSc in
% Computer Science in 1988. Schorn and Nievergelt returned to ETH Zurich in
% 1989, with Schorn eventually completing his PhD in 1991. 
% written in Object Pascal

% \subsection{General algorithm animation}
% 
% \paragraph{BALSA} \cite{brown1984system}
% 
% \paragraph{BALSA-II}
% 
% \paragraph{Zeus}
% 
% \paragraph{AnimA}
% 
% \paragraph{TANGO} \cite{stasko1990tango}
% 
% \paragraph{POLKA} \cite{stasko1995polka}
% 
% \paragraph{SAMBA} \cite{stasko1995samba}
