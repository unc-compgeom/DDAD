%==============================================================================
% @author Clinton Freeman <freeman@cs.unc.edu>
% @date 2014-12-28
%==============================================================================

\FloatBarrier
\section{Case Study: Gift Wrapping the Integer Hull}
\label{sec:case-ihull}

In this section, we use the workbench to implement and visualize gift wrapping
the integer hull of a convex polygon. The algorithm takes as input a
convex polygon $P$ and produces as output the convex hull of integer points
inside $P$. We conclude by using the GUI to generate an input polygon and run
the algorithm.

%==============================================================================

\subsection{Algorithm Overview}

In this section we first define invariants for identifying tangent ranges and
vertices of the integer hull and derive correct but slow code for maintaining
them. Next we derive the initialization and finalization.  Finally, we modify
the invariants and the code to compute the same sequence of hull vertices
faster, and analyze the running time.

Our input is a convex polygon, which can be represented as a counter-clockwise,
circular sequence of vertices and oriented supporting lines through the edges.
Our algorithms keep track of a single {\it current edge} with its support line 
$L$ that has the polygon to its left.  The current edge includes its ccw vertex,
denoted the {\it current vertex} $P$, but does not include the cw vertex.  The
operation $[P,L]=\op advance(P)$ advances to the next polygon vertex and line in
ccw order.

The output will be a ccw sequence of vertices of the integer hull, possibly with
a prefix of some additional vertices outside the integer hull.  As we will see
later, the precise statement is that the output is a sequence of vertices of the
integer hull of the region to the left of all polygon lines seen thus far---by
wrapping twice and discarding the prefix, we avoid specialized code for
initialization. 

% We begin with the invariants and body of a simple, but slow,
% algorithm~\ref{alg:slow} that identifies the sequence of integer hull vertices
% whose tangents lie in wedges on a stack.  It uses divide and conquer on wedges:
% either it identifies the hull vertex $z$ whose tangent range includes the entire
% wedge at the top of the stack, advancing to a new hull vertex if necessary,
% or it splits the wedge and stacks the pieces.
% 
% The number of invariants is almost larger than the number of lines of the
% algorithm, so let's start simply with the stack invariant.
% \begin{lemma}\label{lem:stack}
% The stack contains wedges defined by pairs of integer vectors $u$,$v$ that
% form unit area parallelograms that are consecutive in ccw order top to
% bottom, provided that this is true initially.
% \end{lemma}
% \begin{proof} 
% Suppose that wedge $[u,v)$ is popped in line~\ref{a1:pop}; if nothing is pushed,
% the invariant holds.  Parallelogram $uv$ has unit area, so $\determ{u.x&u.y\\
% v.x&v.y}=1$.  By row operations $\determ{u.x&u.y\\ u.x+v.x&u.y+v.y}=1$, so
% parallelogram $u,u{+}v$ also has unit area, as does $u{+}v,v$.  These are pushed
% onto the stack (line~\ref{a1:push}) in an order that preserves the invariant.
% \end{proof}
%  
% \begin{algorithm}[ht]
%   \begin{minipage}{\algwidth}\hrule\medskip
% Input: 
% $S$, a stack of wedges defined by pairs of integer vectors $u$,$v$ that
% form unit area parallelograms. From top to bottom the wedges are
% consecutive in ccw order. \\
% $P$,$L$ the current vertex and line of a convex polygon, and\\
% $z$, a vertex of the integer hull with a supporting line direction in the top wedge
% $u$,$v$ and with segment $\seg{z,u}$ crossing the current polygon edge.\\
% Output: The sequence of integer hull vertices whose tangent ranges
% cover the wedges on the stack. 
%  \smallskip
% \begin{alg}%
% proc \op ProcessWedge(S,z, P,L)\+\\
% \kw{while} stack $S$ is not empty\+\\
% $[u,v)=\op pop(S)$;\label{a1:pop}\\
% \kw{if} $L$ parallels or intersects \ray{z,v}\+\\
% \kw{while} $P$ is right of \ray{z,v} \label{a1:advP}\+\\
% $[P,L]=\op advance(P)$;\\
% \kw{if} $z$ right of $L$, return $\emptyset$; \kw{endif} \label{a1:test}\-\\
% \kw{endwhile}\\
% $z=\op lastBefore(z,v,L)$;\label{a1:z}\-\\
% \kw{else} split wedge\+\\
% \op push([u+v,v),S);\label{a1:push}\\
% \op push([u,u+v),S);\-\\
% \kw{endif}\-\\
% \kw{endwhile}\>\>\>\>\>\>\>\qquad\smash{\includegraphics[scale=.9]{figs/wedge}}\stepcounter{figure}
% \end{alg}\medskip\hrule
% \end{minipage}
% \caption{We either find that the
%   tangent range of $z$ covers the rest of the wedge, and possibly
%   identifies the next integer hull vertex ccw,  or we split the wedge}
% \label{alg:slow}
% \end{algorithm}
% 
% 
% Let $z$ be an integer point that satisfies three conditions with
% respect to the wedge $[u,v)$ and current edge of the convex polygon defining the hull 
% \begin{enumerate}[noitemsep]
% \item $z$ has a supporting line direction $\tau$ that lies in wedge
% $[u,v)$.
% \item \seg{z,z{+}u} intersects the current polygon edge, and
% \item $z$  lies on or to the left of all polygon lines seen so far.
% \end{enumerate}
% 
% \begin{lemma}
% Algorithm~\ref{alg:slow} maintains the conditions on $z$ and the
% current polygon edge, or reveals that the polygon contains no grid
% points. 
% \end{lemma}
% \begin{proof}
% Suppose first that the algorithm reaches
% line~\ref{a1:advP} because current polygon line $L$ intersects
% \ray{z,v}.  The polygon boundary from $\ray{z,u}$ to $\ray{z,v}$
% remains inside a sequence of unit parallelograms in direction
% $v$,  as seen in the figure with Algorithm~\ref{alg:slow}.   We
% advance $P$ until the current edge intersects \ray{z,v}.  
% 
% In the process of advancing, if we find $z$ to the right of the
% current polygon line $L$, then the entire polygon must lie on and to
% the left of $L$, support line $\tau$, and the boundary from $u$ to
% $\ray{z,v}$.  Since this is entirely contained in the unit
% parallelograms, the polygon contains no integer points, and we return
% and report that.  
% 
% In the usual case, by line~\ref{a1:z} we will have verified that $z$ remains left or on all lines
% that we advanced over, and we can find the last integer point on
% \ray{z,v} inside the  polygon, and call it $z$.  Whether this is a new
% $z$ or the old $z$, the current edge of the polygon intersects
% $\seg{z,v}$, which will become $\seg{z,u}$ after the next pop, by the
% stack invariant of Lemma~\ref{lem:stack}. Also, there will be a
% support line just ccw of $v$ which will serve as~$\tau$.
% 
% On the other hand, if algorithm reaches
% line~\ref{a1:push} then the only concern is if the  support line
% $\tau$ does not lie in the top wedge $u,u{+}v$ because it lies in
% $u{+}v,v$ instead.  But that can happen only if $z+u{+}v$ does not lie
% inside the polygon, so the top wedge will be popped on the next
% iteration.  (We could have handled this by pushing only if needed, but
% that will happen in the next algorithm.)
% 
% Thus, the conditions on $z$, the current edge of the polygon, and top
% wedge are invariant. 
% \end{proof}
% 
% 
% \subsection{Initialization and finalization}
% To initialize Algorithm~\ref{alg:slow}, we 
% push onto the stack six quadrants, in order II,I,IV,III,II,I, so I is
% on top with $u=(1,0)$ and $v=(0,1)$.  We choose coordinates for $z$
% as floor of the maximum $x$ and minimum $y$ coordinates of the
% polygon, and let the current $P$ be the vertex with max $x$
% coordinate and create an initial line $L$ parallel to the $y$ axis
% through $P$.  These choices satisfy all the invariants. 
% 
% We claim that this algorithm does terminate; it cannot keep filling the stack without advancing $P$. 
% Consider a wedge $[u,v)$ for which $L$ does not cross \ray{z,v}.  Note that this puts $P$ right of \ray{z,v}.  
% Since $L$ does separate $z$ from $z+u$, $L$ and \ray{zP} define a non-empty angle range around $P$.
% Split the wedge and consider whether $u+v$ lies cw, in, or ccw of this range: if cw, we continue in $[u,u{+}v)$, if ccw, we push and pop $[u,u{+}v)$  and continue in $[u{+}v, v)$.  In both cases this is the range containing $P$ in the interior.  Because the wedge narrows, we eventually  find $u+v$ in the range between $L$ and \ray{zP}, continuing in a wedge containing $P$ in the interior, so we advance $P$.   
% 
% Note from this argument that  rather than coarser parameters like 
% It is tempting to try use $n$, the number of vertices of
% the input polygon, and $d$, its diameter, to establish a coarse bound
% of $O(n+d)$, but the argument shows that the number of times a wedge is split 
% depends on the detailed geometry of the input, like the slope of polygon line $L$.
% in fact, splitting can generate vectors that extend  far beyond the polygon if
% the precision of the coordinates defining $L$ is sufficiently high.
% This is one of the problems that we will solve in the next subsection.
% 
% We mentioned that this initialization can generate a prefix of integer
% points 
% that lie outside the final integer hull. Our initial $z$ certainly
% does, since it is outside the polygon.  The next point may also,
% because if we split the polygon into upper and lower chains at the
% vertices that are extreme in $x$, then we advance $P$ on the upper
% chain to cross \ray{z,v} and take the highest grid point below the
% upper chain as our new $z$.  But this $z$ may also be below the lower chain; a
% thin polygon 
% may contain no integer points at this $x$ coordinate.  Our $z$ is
% known to be inside only those lines that have been seen.  When we
% complete quadrant II and pop III, however, we have seen all upper lines, and all
% lower lines relevant to determining the integer hull vertex with minimum
% $x$ coordinate, breaking ties by min $y$.  We can then discard all
% preceding vertices and continue.  When we complete II for the second
% time, we have returned to the same vertex and completed the integer
% hull.   We summarize in the following theorem.
% \begin{theorem}
% Algorithm~\ref{alg:slow} computes the vertices of the integer hull
% inside the given convex polygon.
% \end{theorem}
% 
% \subsection{Wrapping the integer hull efficiently}
% 
% Algorithm~\ref{alg:slow} can be slow for two reasons.  The first is
% subtle, but was already mentioned: since $L$'s
% slope comes from the input polygon unnecessary refinement may be
% needed to find a ray \ray{z,v} that intersects line $L$.  
% The second is the reason
% that Euclid's subtractive GCD computation can be slow:  If we
% alternate between replacing $u$ or $v$ with $u+v$, then vector length
% grows exponentially, but if we update just one, then length grows
% linearly.  Just as we
% would prefer to divide instead of performing repeated subtraction for GCD, we
% would prefer to multiply instead of performing repeated addition to find the next wedge. 
% 
% 
% Figure~\ref{fig:wedge2} copies the parallelogram $uv$ in both the $u$ and $v$ directions, and considers how the polygon boundary can next exit this {\it extended wedge}. Let $z'=z+u+v$.
% \smallbreak\noindent{\bf A:\/} The polygon could reach segment \seg{z,v}, with $z$ continuing to be the integer hull vertex whose tangent range we are finding, as in Alg.~\ref{alg:slow}.
% \smallbreak\noindent{\bf B:\/}   The polygon could reach ray
% \ray{z,v}, completing the tangent range for $z$, and beginning one for new hull vertex $z$,
% as in Alg.~\ref{alg:slow}.
% \smallbreak\noindent{\bf C:\/}  The polygon could reach ray \ray{z',v}, which corresponds to repeating $[u,v)\to [u{+}v,v)$ by repeated
% splits and pops in Alg.~\ref{alg:slow}.  We update $u=\op lastBefore(z',v,L)-z$, jumping straight to the outcome of the repeated splits and pops.
% \smallbreak\noindent{\bf D:\/}  The polygon could reach ray \ray{z',u}, which corresponds to repeated splitting $[u,v)\to [u,u{+}v)$ in Alg.~\ref{alg:slow}.
% We update $v=\op lastBefore(z',u,L)-z$, but need to change the  invariants since we don't want to push all the $[u{+}v,v)$ wedges that we skip over.  
% 
% \figbox[l]{\includegraphics[width=.5\textwidth]{figs/wedge2}}{figure}{fig:wedge2}{Exiting $uv$ parallelograms in Alg.~\ref{alg:fast}}
% \smallbreak\noindent{\bf E:\/}  The polygon line could have $z$ to the right, in which case the polygon contains no grid points as in Alg.~\ref{alg:slow}.
% 
% 
% Case D necessitates the following invariant changes.  Before updating
% $v$, in line~\ref{a2:D} we push $[u,v)$ to represent all wedges
%   $[iu+v,(i+1)u+v)$, which is just the area covered by copying the
%     parallelogram $uv$ in direction $u$.  Thus, the stacked wedges may
%     be nested or consecutive, but they will be in ccw order by $v$.
%     Since popping a wedge may give a containing wedge, we no longer
%     know that \seg{z,u} intersects the current polygon edge, but we do
%     know that it intersected the polygon at the time it was pushed,
%     and that the current edge has a point inside the wedge.  With
%     these invariant changes, we can show that Alg.~\ref{alg:fast}
%     identifies the same vertices as Alg.~\ref{alg:slow}, so with the
%     same initialization and finalization it computes the integer hull.
% 
% \begin{theorem}
%   Given an $n$-vertex convex polygon of diameter $d$ with single
%   precision coordinates, Algorithm~\ref{alg:fast} computes the
%   $m$-vertex integer hull in $O(n+m\log d)$ steps and double
%   precision. Space required is $O(\log d)$.
% \end{theorem}
% \begin{proof}
% Advancing $P$ takes $O(n)$ time, since we go around less than twice.
% Cases C and D alternate until terminated by Case A, B, or E. Case E
% can happen only once; B only $m$ times.  Since C takes $O(1)$ time and
% does not increase stack height, we fold it into the case that follows.
% Each string of C/D terminates with a case B, creating a new vertex of
% the integer hull, or E; the vectors grow at least as large as the
% Fibonacci numbers~\cite{davenport1999higher}, and so have at most
% $\log_\phi d$ alternations for the new edge, where $\phi=(1+\sqrt
% 5)/2$ is the golden ratio.  Since case D is the only case that pushes
% more than one item onto the stack after initialization, the stack
% depth is $\log_\phi d$, and the total number of wedges stacked is
% $m\log_\phi d$.  This also bounds the number of wedges popped by
% $A$.
% \end{proof}
% With a little more work and Jensen's inequality, we can express the
% time bound as $O(n+m\log (d/m))$, but since $m=O(d^{2/3})$ (Har-Peled
% has a short proof~\cite{har1998output}) this would not be an
% asymptotic improvement.  The same worst-case time bound can be
% achieved by a constant-space algorithm -- rather than stacking the
% wedges, we could start from a quadrant and work forward each time.
% Our preliminary experiments suggest that stacking is faster; we will
% include a brief experimental section in the next version of the paper.
% 
% 
% \begin{algorithm}[htp]
%   \begin{minipage}{\algwidth}\hrule\medskip
% Input: 
% $S$, a stack of wedges defined by pairs of integer vectors $u$,$v$ that
% form unit area parallelograms. From top to bottom the wedges are
%  in ccw order by $v$ and consecutive wedges are either adjacent or nested. \\
% $z$, an integer point, left or on all polygon lines seen so far, with a tangent in the top wedge
% $u$,$v$, with segment $\seg{z,u}$ crossing the polygon.\\
% $P$,$L$ the current vertex and line of a convex polygon, for which
%  some point of the current edge is right of \ray{z,v} and \ray{z{+}v,u}.\\
% Output: The sequence of integer hull vertices whose tangent ranges
% cover the wedges on the stack. 
%  \smallskip
% \begin{alg}%
% proc \op ProcessWedges(S,z,P,L)\+\\
% \kw{while} stack $S$ is not empty\+\\
% $[u,v)=\op pop(S)$;\label{a2:pop}\\
% \kw{while} $P$ is right of \ray{z,v} and \ray{z{+}v,u}\+\\
% $[P,L]=\op advance(P)$;\label{a2:advP}\\
% \kw{if} $z$ right of $L$, return $\emptyset$; \kw{endif} \-\\
% \kw{endwhile}\\
% \kw{if} $z+u$ is not right of $L$ \label{a2:A}// if not Case A\+\\
% \kw{if} $z+u+v$ is not right of $L$\+\\
% \op push([u,v),S); \label{a2:D}// Case D: in $u$ direction\\
% $v=\op lastBefore(z{+}v,u,L)-z$;\\
% \op push([u,v),S);\-\\
% \kw{else} // in $v$ direction\+\\
% \kw{while} $P$ is right of \ray{z,v} and left of \ray{z{+}u,v}\label{a2:while2}\+\\
% $[P,L]=\op advance(P)$;\label{a2:advP2}\\
% \kw{if} $z$ right of $L$, return $\emptyset$; \kw{endif} \-\\
% \kw{endwhile}\\
% \kw{if} $P$ is not right of \ray{z,v}  // Case B (or A): end wedge\+\\
% $z=\op lastBefore(z,v,L)$; \label{a2:B}// new (or same) vertex\-\\
% \kw{else} // Case C:  \+\\
% $u=\op lastBefore(z{+}u,v,L)-z$;\label{a2:C}\\
% \op push([u,v),S);\-\\
% \kw{endif}\-\\
% \kw{endif}\-\\
% \kw{endif}\-\\
% \kw{endwhile}
% \end{alg}\medskip\hrule
% \end{minipage}
% \caption{Faster processing for the same hull vertices}
% \label{alg:fast}
% \end{algorithm}

%==============================================================================

\subsection{Algorithm Implementation}

%==============================================================================

\subsection{Generating Input Data and Executing IHull} 
