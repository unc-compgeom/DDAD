\FloatBarrier
\section{DDAD Workbench Desiderata}

Geometric objects on a digital computer are composed of two types of data:
numerical and combinatorial. Examples of numerical data include the
Cartesian coordinates of a point in 3-space, the length of a line segment
connecting two such points, or the angle between two such line segments.
Examples of combinatorial information include grouping two points as an
edge, grouping a collection of edges as a face, or grouping a collection of
faces as a surface.

Geometric algorithms that operate on geometric objects are best thought of as
two types of operations: predicates and constructions. Predicates determine
relationships between objects. A predicate might determine if a point is to
the left, right, or is collinear with a line segment, determine if a point is
inside, outside, or on a circle, or determine if a line intersects a plane in
one, none, or infinitely many points. Constructions produce new geometric
objects from existing geometric objects. A construction might produce the
rotation of a point around an origin, produce the point of intersection
between two line segments, or produce an offset of an algebraic curve.

\paragraph{Scope} describes the range of algorithms the system can visualize.
First, the system should visualize geometric algorithms implemented with
primitives and predicates from the DDAD library. Second, the system should
provide example implementations and visualizations of algorithms that solve
textbook computational geometry problems. Third, the system should be flexible
enough to visualize new geometric algorithms from the facilities used in example
visualizations. Fourth, the system should run well given inputs large enough to
verify the correctness of a particular implementation.

\paragraph{Content} describes what subset of information about the software is
visualized by the SV system. For algorithm implementers, the system should
visualize information about the concrete implementation of the algorithm. This
entails combining traditional debugging facilities with the new ability to
visualize geometric structures and operations. For algorithm presenters, the
system should visualize information about the high-level algorithm description.
This entails mechanisms for displaying visualizations that do not necessarily
correspond to the underlying implementation details.

\paragraph{Form} describes the characteristics of the output of the system (the
visualization). First, the output medium should be an interactive computer
program. This is sufficient for implementers, but presenters may want to produce
a digital recording of an algorithm animation. Fortunately, screen capture
programs can facilitate producing this type of output. Second, the output should
have a rich graphical vocabulary, displaying standard graphics primitives
(points, lines, polygons), each with a variety of attributes (color,
transparency, pattern, texture). Presenters in particular desire highly
expressive output. Third, the output should contain both 2D and 3D elements.
The third dimension can be used both for visualizing inherently 3D algorithms
and to effectively display non-dimensional information. Fourth, the output
should have multiple levels of information granularity, with the ability to
elide and reveal specific levels as needed. Fifth, users should be able to view
the same data in multiple ways. This capability is particularly pertinent in
geometry where insight into a problem may be gained from viewing a dual
formulation (\emph{e.g.} viewing points as lines).

\paragraph{Method} describes how the visualization is specified. Implementers
desire highly automated specifications that minimally invade implementation source
code. Increased automation increases the likelihood a tool will be used for
debugging assistance. Useful automated visualizations require a high level of
intelligence in order to correctly recognize and display high level data
structures. In contrast, presenters are less concerned with automated
visualization specifications, and accept a higher degree of invasiveness in
return for customization and increased expressibility. The system should strike
a balance between automation and customization.

\paragraph{Interaction} describes how the user of the software visualization
system interacts with and controls it. First, users should be able to control
the speed of execution. This includes starting, stopping, speeding up, slowing
down, and single stepping. Second, users should be able to control a camera to
navigate around the scene and to focus on different objects. This is especially
important for 3D algorithms and larger data sets. Third, users should be able to
record and play back particular runs of a visualization. For presenters this
could aid pedagogy, and implementers might record when an algorithm fails.

\subsection{Current Capabilities}

The workbench currently satisfies the desired functionality as follows. It
extends the DDAD library with visual types and encapsulates these into a
geometry kernel. The geometry kernel is paired with a visualizer that is capable
of animating algorithm execution. The workbench strikes a balance between
automated visualizations and expressive power by using object
oriented design to encapsulate visual commands in lower level objects. The user
is able to control the speed of execution and pause at important moments. They
control two different views of the scene and can move these views interactively
as the algorithm executes.
