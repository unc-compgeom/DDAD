%==============================================================================
% @author Clinton Freeman <freeman@cs.unc.edu>
% @date 2014-05-23
%==============================================================================

\FloatBarrier
\section{Case Study: Melkman's Algorithm}

Melkman's algorithm computes the convex hull of a simple polyline in $O(n)$
time~\cite{melkman1987line}. It operates by using the 2D orientation predicate.

Animating any algorithm begins with generating the appropriate input. In the
case of Melkman's algorithm, we begin by creating a simple polyline. This is
accomplished by the user clicking on points in the 2D top-down orthographic
view. The user may choose to place integer vertex coordinates or turn off
snapping so that vertex coords are floating point or rational coords.

\begin{mdframed}[linecolor=white, backgroundcolor=algback, frametitle={Algorithm
Melkman}] \begin{algorithmic}[1]    
    \Require Simple polyline $P = \langle v_1, \ldots, v_m \rangle$.
    \Ensure $\text{CH}(P)$.
    \vspace{0.75em}
    \Procedure{Melkman}{$P$}
    \EndProcedure
\end{algorithmic}
\end{mdframed} 

\begin{lstlisting}
Polygon_2r Melkman(const PolyChain_2r& P, Visual::IGeometryObserver* ge_obs) {
    Polygon_2r CH_P;


    return CH_P;
}
\end{lstlisting}