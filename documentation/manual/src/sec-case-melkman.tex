%==============================================================================
% @author Clinton Freeman <freeman@cs.unc.edu>
% @date 2014-05-23
%==============================================================================

\FloatBarrier
\section{Case Study: Melkman's Algorithm}

This section examines using our workbench to animate Melkman's convex hull
algorithm. By providing a concrete example we hope to illuminate the major steps
required to animate algorithms in general. Before we get started, let us briefly
examine the algorithm in question.

A \emph{polyline} $P$ is a polygonal chain of vertices $p_1, p_2, \ldots, p_n$
connected by line segments $p_ip_{i+1}$ for $1 \leq i < n$. $P$ is \emph{simple}
if the only intersection between segments is at their shared endpoints.
Melkman's algorithm incrementally computes the convex hull of a simple polyline
in $O(n)$ time~\cite{melkman1987line}. It can also be used to compute the convex
hull of arbitrary point sets if we first sort by $x$ coordinate, breaking ties
by $y$ coordinate.


% Incremental convex hull algorithms construct the hull by examining each input
% point in turn, exploiting structure in the partial hull to help reduce
% computation. 
% Graham and Yao observed that if we know the points in advance, we
% may make our task easier by considering them in sorted order by $x$ coordinate,
% breaking ties by $y$ coordinate. Then point $p_i$ will always be a vertex of the
% convex hull $\text{CH}(P_i)$, and it will either be adjacent to $p_{i-1}$ or
% will cause $p_{i-1}$ to be removed from $\text{CH}(P_i)$.



\begin{mdframed}[linecolor=white, backgroundcolor=algback, frametitle={Algorithm
Melkman}] \begin{algorithmic}[1]    
    \Require Simple polyline $P = \langle v_1, \ldots, v_m \rangle$.
    \Ensure $\text{CH}(P)$.
    \vspace{0.75em}
    \Procedure{Melkman}{$P$}
    \State $\text{CH}(P).push(v_2);$ $\text{CH}(P).push(v_1);$
    $\text{CH}(P).push(v_2);$ \Comment{Init hull}
    \For{$i=3\ldots m$}
    	\State derp 
    \EndFor
    \EndProcedure
\end{algorithmic}
\end{mdframed} 













Animating an algorithm using the workbench is composed of a number of tasks,
namely 
\begin{itemize}
  \item Implement input data structures and instrument them with visualization
  code.
  \item Optionally modify the GUI to allow the user to create instances of
  the input data structure.
  \item Implement output data structures and instrument them with visualization
  code.
  \item Implement predicates.
  \item Implement the algorithm and instrument it with a small amount of
  visualization code.
  \item Optionally modify the GUI to allow the user to run the algorithm on
  selected input data. 
\end{itemize}

\subsection{Data Structures}

polychain\_2r

polygon\_2r

\subsection{Generating Input Data}




Create button and implement click handler. buttongroup to model mutually
exclusive input states.

define input states. configmanager singleton. 



Handle ortho widget mouse clicks
forward signals from ortho to scene observer

\subsection{Predicates}

Given an oriented line $pq$ and a point $r$, the 2D orientation predicate
$\textsc{Orient2D}(p, q, r)$ answers the question, ``is $r$ to the left, right,
or on $pq$?'' It is often written as the sign of the 2-by-2 determinant, $$
\textsc{Orient2D}(p, q, r) = \textsc{Sign}\left( \begin{vmatrix} p_x-r_x &
p_y-r_y \\ q_x-r_x & q_y-r_y \end{vmatrix} \right).$$












\subsection{Code Listing}

\lstinputlisting{code-samples/melkman.cpp}


































% Animating any algorithm begins with generating the appropriate input. In the
% case of Melkman's algorithm, we begin by creating a simple polyline. This is
% accomplished by the user clicking on points in the 2D top-down orthographic
% view. The user may choose to place integer vertex coordinates or turn off
% snapping so that vertex coords are floating point or rational coords.
% 
% 
% 
% \begin{lstlisting}
% Polygon_2r Melkman(const PolyChain_2r& P, Visual::IGeometryObserver* ge_obs) {
%     Polygon_2r CH_P;
% 
% 
%     return CH_P;
% }
% \end{lstlisting}