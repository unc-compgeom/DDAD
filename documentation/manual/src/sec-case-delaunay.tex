\FloatBarrier
\section{Case Study: Incremental Delaunay Triangulation}

This section reviews animating an incremental Delaunay triangulation algorithm.
The algorithm is covered extensively in \cite{de2000computational}.  

\subsection{Algorithm Overview}

The algorithm uses the quadedge data structure~\cite{guibas1985primitives}.

\begin{mdframed}[linecolor=white, backgroundcolor=algback, frametitle={Algorithm
Delaunay}] \begin{algorithmic}[1]
    \Require A set $P$ of $n+1$ points in the plane.
    \Ensure A Delaunay triangulation of $P$.
    \vspace{0.75em}
    \Procedure{Delaunay}{$P$}
    \State Let $p_0$ be the lexicographically highest point of $P$.
    \State Let $p_{-1}, p_{-2}$ be two points far away such that $P$ is
    contained in the triangle $p_0p_{-1}p_{-2}$.
    \State Initialize $\mathcal{T}$ as the triangulation consisting of the
    single triangle $p_0p_{-1}p_{-2}$.
    \State Compute a random permutation $p_1, p_2, \ldots, p_n$ of $P /
    \{p_0\}$.
    \For{$r = 1 \ldots n$}
   \State derp
    \EndFor \EndProcedure
\end{algorithmic}
\end{mdframed} 

\subsection{Code Listing}

\lstinputlisting{code-samples/delaunay.cpp}

\subsection{Generating Input Data}

The user is given a button in the GUI and may browse to read in point sets
stored in text files.
