\FloatBarrier
\section{Case Study: Incremental Delaunay Triangulation}
\label{sec:case-delaunay}

% The DDAD workbench makes it easy for presenters and implementers to quickly
% visualize new geometric objects and algorithms. In this section, we overview
% using the workbench to implement a bare-bones visualization of Melkman's convex
% hull algorithm~\cite{melkman1987line}. First, we implement the basic data
% types used by the algorithm and augment their methods with visualization
% code. Second, we use these data types to implement Melkman's algorithm. 
% By visualizing the data types in an object-oriented way, we arrive at a
% clean implementation of the algorithm.

The DDAD workbench can quickly visualize three-dimensional algorithms. In this
section, we use the workbench to visualize an implementation of incremental
delaunay triangulation. The presentation follows the same format as our previous
case study. First, we review the basic data types used by the algorithm and
show how to augment their methods with visualization code. Second, we use these
data types to implement the algorithm.

%\subsection{Data Type Design and Implementation}



% This section reviews animating an incremental Delaunay triangulation algorithm.
% The algorithm is covered extensively in \cite{de2000computational}.  

%\subsection{Algorithm Overview}

% The algorithm uses the quadedge data structure~\cite{guibas1985primitives}.
% 
% \begin{mdframed}[linecolor=white, backgroundcolor=algback, frametitle={Algorithm
% Delaunay}] \begin{algorithmic}[1]
%     \Require A set $P$ of $n+1$ points in the plane.
%     \Ensure A Delaunay triangulation of $P$.
%     \vspace{0.75em}
%     \Procedure{Delaunay}{$P$}
%     \State Let $p_0$ be the lexicographically highest point of $P$.
%     \State Let $p_{-1}, p_{-2}$ be two points far away such that $P$ is
%     contained in the triangle $p_0p_{-1}p_{-2}$.
%     \State Initialize $\mathcal{T}$ as the triangulation consisting of the
%     single triangle $p_0p_{-1}p_{-2}$.
%     \State Compute a random permutation $p_1, p_2, \ldots, p_n$ of $P /
%     \{p_0\}$.
%     \For{$r = 1 \ldots n$}
%    \State derp
%     \EndFor \EndProcedure
% \end{algorithmic}
% \end{mdframed} 

% \subsection{Code Listing}
% 
% \lstinputlisting{code-samples/delaunay.cpp}
% 
% \subsection{Generating Input Data}
% 
% The user is given a button in the GUI and may browse to read in point sets
% stored in text files.
