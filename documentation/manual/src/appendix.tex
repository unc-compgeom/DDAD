\FloatBarrier
\section{Appendix}

This section contains two types of information. First, there is
information about aspects of the workbench that should help users and
contributors understand its inner workings. Second, there is information that
can serve as background for some of the concepts found in the main paper. Much
of this section is mirrored on the DDAD wiki; areas for future work are captured
in issues on Github. References to issues are labeled with a triangle
($\triangle$) and the issue number.

%==============================================================================

\subsection{Adding New Geometric Types}\label{appdx:adding-new-geometric-types}

This section covers how to add new geometric types to the DDAD workbench.

\begin{enumerate} 
  \item Create a new header and implementation file under the geometry library
  folder. See folder structure (\ref{appdx:folder-structure}). File names are in
  lower case with underscores between words (e.g. \texttt{geometry/my\_geometric\_type.h}.)
  Smaller, closely related items should be placed in the same file to avoid
  proliferation of tiny fragment files (e.g. polyline and polygon types are both placed into
  files named \texttt{polygon.h/polygon.cpp.}) See the coding
  guidelines (\ref{appdx:coding-guidelines}) for more detail on naming
  conventions.
  \item Define the interface of your new geometric type. You may use the
  template provided in listing~\ref{geometric-type-template} or look to existing
  geometric types for guidance.
  Importantly, you must inherit from \texttt{Visual::Geometry}. Some things you
  should notice about the template:
  \begin{itemize}
    \item All files start with the license.% (\ref{appdx:license}). 
    \item Next is a short one-liner describing the file that uses the
    \href{http://www.stack.nl/~dimitri/doxygen/}{Doxygen} tag \texttt{@brief}.
    \item All header files use
    \href{http://en.wikipedia.org/wiki/Include\_guard}{include guards}. Include
    guards in the geometry project are prefixed with \texttt{GE\_} and workbench
    include guards are prefixed with \texttt{WB\_}. Next, add the name of your
    file in uppercase. All guards end with \texttt{\_H}.
    \item All files include \texttt{common.h} which includes system utilities
    and logging. Visual types include \texttt{visual.h} which defines the visual
    geometry interfaces.
    \item Geometric types are suffixed with their dimension and the underlying
    arithmetic type. For example, \texttt{\_2f} denotes a 2 dimensional object
    with single-precision floating-point coordinates. See the coding
    guidelines (\ref{appdx:coding-guidelines}) for a full suffix listing.
  \end{itemize}
%   \lstinputlisting[caption=Geometric Type Template,
%   label=geometric-type-template]{code-samples/geometric-type-template.cpp}
  \item Provide implementations of the default constructor, copy constructor,
  and destructor. The copy constructor is important to implement if you plan on
  returning a copy of your type from functions, e.g. \texttt{Melkman} returns a
  \texttt{Polygon\_2r}. If you have not implemented a copy constructor, you may
  experience double-destruction of returned visualization primitives.
  \item Implement your geometric type as you normally would.
  \item Augment the type's methods with visual signals to reflect the state of
  the object. For example, listing~\ref{polyline-push-back} shows that adding a
  point to a polyline also registers the point so it can be viewed.
%   \lstinputlisting[caption=Example of Augmented Method,
%   label=polyline-push-back]{code-samples/polyline-push-back.cpp}
  \item Define a scene object (\ref{appdx:scene-objects}) type in
  \texttt{workbench/scene.h} that corresponds to your geometric type. You may
  use the template in listing~\ref{scene-object-template}.
%   \lstinputlisting[caption=Scene Object Template,
%   label=scene-object-template]{code-samples/scene-object-template.cpp}
  \item Implement the \texttt{Select}, \texttt{Deselect}, and \texttt{Intersect}
  methods so that your object can be selected in the editor. Refer to
  picking (\ref{appdx:picking}) for more details.
  \item Add a button to the Create tab defining ways for the user to create your
  object. Add the button using Qt Designer by opening
  \texttt{workbench/forms/window\_main.ui}. The button should have a
  human-readable name for your object and an icon representation. Add your
  button to the \texttt{buttonGroup} with \texttt{Right-click -> Assign to
  button group}.
  \item  Create a custom \texttt{MyGeometricObjectCreationMethod} widget. Click
  on \texttt{File -> New File or Project}. You will see the dialog shown in
  figure~\ref{fig:creation-method}.
  \begin{figure}[H]
	\centering
	\shadowimage[width=\textwidth]{figures/gui-creation-method-new-file}
	\caption{Dialog for creating a new creation method widget.} 
	\label{fig:creation-method}
  \end{figure}
  \item Implement the \texttt{toggle} event handler for the button. Right-click
  on the button in Qt Designer and choose \texttt{Go to slot...} then
  \texttt{toggled(bool)}. This will automatically add the appropriate member
  functions onto the main window class and navigate you to the implementation
  stub in \texttt{qt\_window\_main.cpp}. Listing~\ref{creation-toggle-handler}
  shows the toggled handler for \texttt{Polyline\_2r} objects. Some things to notice:
%   \lstinputlisting[caption=Creation Button Toggle Handler,
%   label=creation-toggle-handler]{code-samples/creation-toggle-handler.cpp}
  \begin{itemize}
    \item All creation buttons must call \texttt{uncheckInputModeButtons()}
    since creation and input mode buttons are mutually exclusive.
    \item Use a static function variable for the \texttt{creation\_method}
    QWidget. The first time the button is \texttt{toggled(true)}, we create the
    widget and we destroy the widget when we are \texttt{toggled(false)}.
    \emph{Be sure to delete the widget to avoid a memory leak.}
    \item We set the input mode so that click events in the Orthographic Widget
    will be handled appropriately.
  \end{itemize}
\end{enumerate}

\lstinputlisting[float, caption=Geometric Type Template,
  label=geometric-type-template]{code-samples/geometric-type-template.cpp}

\lstinputlisting[float, caption=Example of Augmented Method,
  label=polyline-push-back]{code-samples/polyline-push-back.cpp}
  
\lstinputlisting[float, caption=Scene Object Template,
  label=scene-object-template]{code-samples/scene-object-template.cpp}
  
\lstinputlisting[float, caption=Creation Button Toggle Handler,
  label=creation-toggle-handler]{code-samples/creation-toggle-handler.cpp}
%==============================================================================

\FloatBarrier
\subsection{Coding Guidelines}\label{appdx:coding-guidelines}

Many organizations and individuals have taken the time to author style
guidelines for the C++ language. A thorough style document is an undertaking in
its own right; they may span many pages and include detailed rules, exceptions,
and justifications for both. For this project, it made sense to leverage the
work of others by adopting and customizing an existing set of guidelines:
\href{http://google-styleguide.googlecode.com/svn/trunk/cppguide.html}{Google's
C++ style guide}. \emph{Contributors should pay attention to coding style so
that we can keep the code uniform, clean, and easy to read.} 

\paragraph{Highlights.}

\begin{itemize}
  \item Strictly adhere to 80-char line widths.
  \item Use lowercase, underscore separated variable names instead of
  camelCase.
  \item Private member variables should be suffixed with an underscore:
  \texttt{my\_member\_var\_}.
\end{itemize}

\paragraph{Additions or Exceptions.}

\begin{itemize}
  \item Geometric types are suffixed with their dimension and underlying
  arithmetic type.
  \begin{itemize}
    \item \texttt{f} = single-precision floating-point
    \item \texttt{d} = double-precision floating-point
    \item \texttt{i} = MPIR integer
    \item \texttt{r} = MPIR rational
  \end{itemize}
\end{itemize}

%==============================================================================

\FloatBarrier
\subsection{Compiling}\label{appdx:compiling}

\begin{enumerate}
  \item Clone the repository. \texttt{git clone
  https://github.com/unc-compgeom/ddad.git}
  \item Install prerequisite software.
  \begin{itemize}
    \item
    \href{http://www.visualstudio.com/en-us/products/free-developer-offers-vs}{Visual Studio Community 2013}
    \item \href{http://www.cmake.org/download/}{CMake 2.8.12 or later}
    \item \href{http://www.qt.io/download-open-source/}{Qt 5.3 32-bit (Desktop
    OpenGL)}
  \end{itemize}
  \item Compile MPIR in \texttt{dependencies/mpir}.
  \item Open DDAD in QtCreator.
\end{enumerate}

%==============================================================================

\FloatBarrier
\subsection{Deploying}\label{appdx:deploying}

Deploying refers to the process of bundling together the workbench executable
with its dependencies for download by non-developers. Windows users should
perform 4 steps to use the DDAD Workbench.

\begin{enumerate}
  \item Install the MSVC redistributable package.
  \item Download a zip file from the website.
  \item Extract the zip file.
  \item Double-click the executable.
\end{enumerate}

This guide makes no assumptions about what is already installed. We could use an
installer to remove the burden of installing the MSVC package, but keeping
deployments to a simple zip file minimizes the barrier for new developers.

\paragraph{Test Setup.} In order to properly test whether your deployment will
work as expected, it is advisable to simulate the target user's machine and
attempt to follow the installation directions listed below. In particular, since
we assume a blank-slate installation, using VMWare to create a clean install of
Windows 7 x64 Home Premium is an easy way to verify the build and instructions.

Outside of advanced user analysis, assuming a blank slate install is about the
best we can do. In practice, users will have a variety of other software
installed that may cause conflicts with the software required to run the
workbench. These sorts of problems should be logged and documented.

\paragraph{Create the zip.} These instructions use absolute paths; please
substitute your own as appropriate.

\begin{itemize}
  \item Create an empty folder that will house your executable and associated
  DLL files.
  \item Copy the release .exe into the folder. You do not need the manifest
  file.
  \item Go into \texttt{C:/Qt/Qt5.1.1/5.1.1/msvc2010\_opengl/bin}
  \item Copy any required DLLs and paste them next to the .exe in your new
  folder. (If you encounter problems, copy all of them to ensure this step is
  not a source of problems. Once you get everything working, you can prune out
  unnecessary DLLs using Dependency Walker. For the default test application, I
  needed: Qt5Core.dll, Qt5Gui.dll, Qt5Widgets.dll, and all dll's prefixed with
  ``icu.'')
  \item Go into \texttt{C:/Qt/Qt5.1.1/5.1.1/msvc2010\_opengl/plugins/platforms}
  \item Copy qwindows.dll and paste it in \texttt{<your new folder>/platforms}.
  You have to perform this step for all applications, even if you don�t know
  what plugins are or don�t think your application is using any.
  \item Compress your new folder.
\end{itemize}

\paragraph{Find the right redistributable.} Qt packages an appropriate
redistributable installer with itself. Instead of hunting around for it on the
internet, we can provide users with Qt's to install.

\begin{enumerate}
  \item Go into \texttt{C:/Qt/Qt5.1.1/vcredist}
  \item You should see something like \texttt{vcredist\_sp1\_x86.exe}.
  \item Provide this file for users to install, as a separate download.
\end{enumerate}

\paragraph{Final process for end users.} After uploading the zipped installation
folder and the corresponding redistributable provided by Qt, the installation
process on the VMWare Windows 7 x64 machine looks like the following.

\begin{enumerate}
  \item Download \texttt{vcredist\_sp1\_x86.exe}.
  \item Run the installer, using all default settings.
  \item Download the zipped installation folder.
  \item Unzip.
  \item Double-click executable.
\end{enumerate}

\paragraph{Notes.} Getting this to work took me longer than expected due to two
problems. First, I had multiple Qt installations on my development machine,
which must have resulted in me copying the wrong dll�s or using the wrong
version of Qt Creator, or something. Uninstalling all versions of Qt, deleting
the corresponding folders, then installing only the version I wanted to use made
debugging much easier. Second, I had difficulty finding the right version of the
MSVC redistributable package on the web. Using the one provided by Qt helped
tremendously; it worked the first time.

%==============================================================================

\FloatBarrier
\subsection{Managing Dependencies}\label{appdx:dependencies}

The DDAD library and workbench share two dependencies:
\href{http://mpir.org/}{MPIR} and
\href{https://github.com/easylogging/easyloggingpp}{EasyLogging++}. MPIR is a
wrapper around GMP and provides arbitrary precision integer, rational, and
floating-point types. EasyLogging++ is a header-only logging library for C++.

\paragraph{The problem.} Many languages have well-supported dependency
management systems: Bower for JavaScript, npm for node.js, Maven and Gradle for
Java. There is currently no standardized way to do dependency management in C++.
This is due to a variety of reasons, including language implementation
differences (mixed support for language features), platform differences (some
platforms do not support exceptions), and compilation differences (static or
shared? debug or production?).
% \begin{quote}
% 	\begin{itemize}
% 	  \item Implementation differences - C++ is a complicated language, and
% 	  different implementations have historically varied in how well they support
% 	  it (how well they can correctly handle various moderate to advanced C++
% 	  code). So there's no guarantee that a library could be built in a particular
% 	  implementation.
% 	  \item Platform differences - Some platforms may not support exceptions. There
% 	  are different implementations of the standard library, with various pros and
% 	  cons. Unlike Java's standardized library, Windows and POSIX APIs can be quite
% 	  different. The filesystem isn't even a part of Standard C++.
% 	  \item Compilation differences - Static or shared? Debug or production build?
% 	  Enable optional dependencies or not? Unlike Java, which has very stable
% 	  bytecode, C++'s lack of a standard ABI means that code may not link properly,
% 	  even if built for the same platform by the same compiler.
% % 	  \item Build system differences - Makefiles? (If so, GNU Make, or something
% % 	  else?) Autotools? CMake? Visual Studio project files? Something else?
% % 	  \item Historical concerns - Because of C's and C++'s age, popular libraries
% % 	  like zlib predate build systems like Maven by quite a bit. Why should zlib
% % 	  switch to some hypothetical C++ build system when what it's doing works? How
% % 	  can a newer, higher-level library switch to some hypothetical build system if
% % 	  it depends on libraries like zlib?
% 	\end{itemize}
% \end{quote}

\paragraph{Possible solutions.} Below, we discuss possible solutions to managing
dependencies and give pros and cons of each approach.

\begin{enumerate}
  \item Since MPIR and EasyLogging++ both have officially maintained GitHub
  repositories, we could use \texttt{git subtree} to bring the source code into
  our repository. Git will maintain a direct link to each repository, and
  updating each dependency can be accomplished with \texttt{git subtree pull}.
  
  Pros: Makes updating dependencies easy. Does not require special
  initialization after user clones DDAD repository. Cross-platform: user is
  responsible for compiling on their machine.
  
  Cons: Requires user to compile dependencies before compiling DDAD. Increased
  learning curve: developers must be adept at git in case things go awry.
  
%   \begin{table}[h]
%     \begin{tabularx}{\textwidth}{XX}
% 	\toprule
% 	Pros & Cons \\
% 	\midrule
% 	\begin{itemize}
% 	  \item Makes updating dependencies easy
% 	  \item Does not require special initialization after user clones DDAD
% 	  repository
% 	  \item Cross-platform: user is responsible for compiling on their machine 
% 	\end{itemize} &
% 	\begin{itemize}
% 	  \item Requires user to compile dependencies before compiling DDAD
% 	  \item Increased learning curve: developers must be adept at git in case
% 	  things go awry
% 	\end{itemize} \\
%     \bottomrule    
% 	\end{tabularx}
%   \end{table}  

  \item Manually take a snapshot of each dependency and copy the source into our
  repository. This does not involve unusual git commands and there will be no
  link to the original dependency repository.
  
  Pros: Low learning curve: anyone can copy over files. Does not require special
  initialization after user clones DDAD repository. Cross-platform: user is
  responsible for compiling on their machine.
  
  Cons: Requires user to compile dependencies before compiling DDAD. Updating
  dependencies is a manual process. Documenting dependency versions is a manual
  process.

%   \begin{table}[h]
%     \begin{tabularx}{\textwidth}{XX}
% 	\toprule
% 	Pros & Cons \\
% 	\midrule
% 	\begin{itemize}
% 	  \item Low learning curve: anyone can copy over files
% 	  \item Does not require special initialization after user clones DDAD
% 	  repository
% 	  \item Cross-platform: user is responsible for compiling on their machine 
% 	\end{itemize} &
% 	\begin{itemize}
% 	  \item Requires user to compile dependencies before compiling DDAD
% 	  \item Updating dependencies is a manual process
% 	  \item Documenting dependency versions is a manual process
% 	\end{itemize} \\
%     \bottomrule    
% 	\end{tabularx}
%   \end{table} 
 
  \item Manually build a snapshot of each dependency and copy the binaries into
  our repository. This does not involve unusual git commands and there will be
  no link to the original dependency repository. If the user is on supported
  OS's and compiler versions then they do not need to compile dependencies
  themselves. 
  
  Pros: Low learning curve, anyone can copy over files. Does not
  require special initialization after user clones DDAD. Users on supported
  platforms do not need to compile dependencies before compiling DDAD. 
  
  Cons: Not cross-platform: binaries are tied to specific operating systems and
  specific compiler versions. Updating dependencies is laborious: developers
  must compile a new snapshot for all OS and compiler version combinations we
  wish to support. Documenting dependency versions is a manual process.
  
% 	  compiling DDAD
% 	  repository
%   \begin{table}[h]
%     \begin{tabularx}{\textwidth}{XX}
% 	\toprule
% 	Pros & Cons \\
% 	\midrule
% 	\begin{itemize}
% 	  \item Low learning curve: anyone can copy over files
% 	  \item Does not require special initialization after user clones DDAD
% 	  repository
% 	  \item Users on supported platforms do not need to compile dependencies before
% 	  compiling DDAD 
% 	\end{itemize} &
% 	\begin{itemize}
% 	  \item Not cross-platform: binaries are tied to specific operating systems and
% 	  specific compiler versions
% 	  \item Updating dependencies is laborious: developers must compile a new
% 	  snapshot for all OS and compiler version combinations we wish to support
% 	  \item Documenting dependency versions is a manual process
% 	\end{itemize} \\
%     \bottomrule    
% 	\end{tabularx}
%   \end{table}

\end{enumerate}

\paragraph{Our solution.} We chose 2. This way we strike a balance between
remaining cross-platform withou  creating too high of a barrier to entry for new
developers. This is justified as follows:

\begin{itemize}
  \item We wish to keep the number of dependencies low. Thus, we do not expect
  to be adding many more beyond MPIR and EasyLogging++.
  \item MPIR is a stable project that has existed for many years. Although there
  is active development and GitHub repository, there are not major feature
  changes that would significantly change the functionality as far as DDAD is
  concerned.
  \item EasyLogging++ is a single-header library, so updating this dependency is
  just a matter of downloading one file and copying it into the project.
\end{itemize}

%==============================================================================

\FloatBarrier
\subsection{Folder Structure}\label{appdx:folder-structure}

Files are laid out according to the following hierarchy.

\begin{itemize}
  \item \texttt{/geometry} - geometry library, includes linear algebra,
  arithmetic, geometric primitives, intersection tests, data structures and
  algorithms.
  \item \texttt{/workbench} - workbench system built using Qt. Contains two
  subfolders:
  \begin{itemize}
    \item \texttt{/resources} - graphics such as application icon and splash
    screen
    \item \texttt{/forms} - Qt ui files for use with Qt Designer. These allow
    the programmer to visually design various screens in the application.
  \end{itemize}
  \item \texttt{/utility} - contains useful code that may be used in any other
  project. e.g. logging.
  \item \texttt{/documentation} - \LaTeX~source files for various types of
  documentation.
  \item \texttt{/dependencies} - 3rd party libraries, e.g. MPIR.
\end{itemize}

%==============================================================================

\FloatBarrier
\subsection{GUI Overview}\label{appdx:gui-overview}

The Graphical User Interface (GUI) comprises 7 components: the menubar, the
toolbar, the orthographic widget, the perspective widget, the interactive panel,
the statusbar, and the console.

% ![GUI Overview](http://cs.unc.edu/~freeman/DDAD/media/gui-overview.png)
\paragraph{Menubar.} The menubar comprises 3 high-level menus: File, Edit, and
Help. It exists primarily as a scaffold for future functionality.

\begin{itemize}
	\item File. The File menu is currently empty but
	\href{https://github.com/unc-compgeom/DDAD/issues/6}{issue $\triangle$ 6}
	specifies saving and loading of files. To do this, we need a mechanism for
	serializing and deserializing the scene, which is a major undertaking.
    \item Edit. The Edit menu contains one element, Preferences (there are
    issues to add duplicating and copy/paste.) Clicking Preferences will open a
    dialog which is split into 3 major panels. On the left is a list of
    subsections. The list currently contains only Grid Settings. On the right is
    a detailed view of the selected subsection. This detail view is
    parameterized by the list on the left. Finally, the bottom contains 4
    buttons: Default, Cancel, Apply, and Ok. The most interesting of these is
    Default, which requires the application to somehow define defaults. Settings
    \& Defaults bring up the notion of a configuration file (e.g. .ini). While
    currently there are not many settings, the dialog sets forth a framework
    into which developers can add new configurations settings.
    \item Help. The Help menu contains two simple elements. The first is a link
    to the user manual (the wiki). The second is an About dialog that provides
    information about the program version and a link to the project webpage.
\end{itemize}

\paragraph{Toolbar.} The toolbar comprises 4 elements: 3 buttons to control
input mode (Select, Translate, and Rotate), and 1 button to toggle snapping to
grid. The three input mode buttons are mutually exclusive with one another and
with the creation buttons in the interactive panel. The select mode is initially
activated and allows the user to pick one object at a time via left-clicking in
the orthographic and perspective views.
\href{https://github.com/unc-compgeom/DDAD/issues/8}{Issue $\triangle$ 8}
specifies multiple-object selection,
\href{https://github.com/unc-compgeom/DDAD/issues/9}{issue $\triangle$ 9}
specifies the rotation mode, and
\href{https://github.com/unc-compgeom/DDAD/issues/10}{issue $\triangle$ 10}
specifies the translation mode.

\paragraph{Orthographic widget.} The orthographic widget allows users to
precisely create and manipulate objects against the backdrop of an integer grid.
The integer grid has minor and major grid lines. Minor lines are drawn in a
lighter color. Major lines occur after a certain number of minor lines - the
default is 8.

Users are able to zoom in and out and pan around the grid. As the user zooms
out, the grid adapts so that the lines are not drawn too close together. At the
highest zoom level, minor lines are drawn 1 unit apart. After the user zooms out
a ways, the grid adapts to draw minor lines 8 units apart. After more zooming,
the grid adapts to draw minor lines 64 units apart, and so on.

\paragraph{Perspective widget.} The perspective widget allows the user to
control a 3D perspective camera that renders the current scene. Users may change
the camera's orientation and move forward (W), backward (S), left (A), right (D)
up (E), and down (C).

\paragraph{Interactive panel.} The interactive panel comprises three tabs:
Create, Modify, and Compute.

The create tab is open by default. It initially has only a gallery of geometric
objects. The buttons in the gallery are mutually exclusive with one another and
with the input mode buttons in the toolbar. Once the user selects one of the
geometric objects, an additional ``Creation Method'' panel will appear with
additional options. The options are parameterized by the selected object type;
often these methods will overlap (e.g. clicking inside the orthographic view to
create a point set or polyline), but may be unique (e.g. a polytope may have a
user-defined length, width, and height). Each method will be listed in a
dropdown menu at the top of the panel. As the user changes between methods,
method-specific controls will appear below the dropdown. For example, when
``Click'' is selected, explanatory text appears; if you change the selection to
``File,'' then the explanatory text will be replaced with a file chooser and
``Generate'' button.

The modify tab is meant to let the user choose the \emph{selection granularity},
or the level at which operations are applied. For example, a polytope is
composed of vertices, edges, and faces. Each of these, along with the whole
polytope, are selectable levels of granularity. When ``edges'' is selected,
picking operations will choose from only the object's edges, when ``vertices''
is selected, picking operations will choose from only the object's vertices, and
so on. Most of this functionality is not yet implemented, but is specified in
\href{https://github.com/unc-compgeom/DDAD/issues/7}{issue $\triangle$ 7}.

The compute tab is composed of the name of the selected object and a drop down
menu of applicable algorithms. The dropdown of algorithms follows the same logic
as the creation method dropdown, meaning that as you change your selection
relevant parameters will appear below. Once the user is with the algorithm
selection and any parameters, they would like to execute the algorithm. For
this, we provide a set of media controls. In particular, they may click the
``Run'' button which will begin algorithm execution. Once the algorithm is
running, editing controls will be disabled.

\paragraph{Statusbar.} The statusbar provides a way of sending messages to the
user, depending on application context. For example, when the user is creating a
polyline, the application cycles through 2 different states. First, the user
needs to click in the orthographic view to position the initial vertex, so we
might simply display: ``Left-click in the orthographic view to place first
vertex.'' Second, the user can either place additional vertices in the same
fashion, or terminate the polyline by right-clicking; the status bar should be
updated to reflect the user's options: ``Left-click again to add more vertices,
or right-click to terminate polyline.'' Finally, after the user right-clicks to
terminate, we transition back to creating a new polyline, so the message should
update as well.

The main window defines a slot for updating the statusbar message (\texttt{void
onUpdateStatusBarMsg(const QString\& status)}). To trigger the handler from
another QObject, you must ensure that the instance is connected to the main
window, then emit an appropriate signal. For more information, see Qt's
documentation on
\href{http://qt-project.org/doc/qt-4.8/signalsandslots.html}{signals and slots} and
\href{http://qt-project.org/doc/qt-4.8/qstatusbar.html}{QStatusBar}.

\paragraph{Console.}

The console is designed to provide non-developers with useful information in
case they experience unexpected (but non-terminal) behavior. By logging to the
console, developers can ask these users to inspect the console to understand the
application behavior. The console is not meant as a replacement of the IDE
console. By default, the console is hidden from view, but may be expanded at any
time.

%==============================================================================

\FloatBarrier
\subsection{Predicates}\label{appdx:predicates}
 
Given three points in the plane, $p$, $q$, and $r$, the 2D orientation predicate
$\textsc{Orient2D}(p, q, r)$ answers the question, ``is $r$ to the left, right,
or on the oriented line $pq$?'' We may write it as the sign of a 2-by-2
determinant, where negative is left, positive is right, and zero is on. In
particular, we have 
\[
\textsc{Orient2D}(p, q, r) = \textsc{Sign}
\left(
	\begin{vmatrix} 
		x_p-x_r & y_p-y_r \\ 
		x_q-x_r & y_q-y_r 
	\end{vmatrix} 
\right).
\]
 

Given 4 points in the plane, $p, q, r$, and $s$, the predicate
$\textsc{InCircle}(p, q, r, s)$ answers the question, ``is $s$ inside, outside,
or on the oriented circle through $p, q$, and $r$?'' We may write it as the sign
of a 4-by-4 determinant, where negative is inside, positive is outside, and zero
is on. In particular, we have

\[ 
\textsc{InCircle}(p, q, r, s) = \textsc{Sign}\left(
\begin{vmatrix}
	x_p & y_p & x_p^2 + y_p^2 & 1 \\
    x_q & y_q & x_q^2 + y_q^2 & 1\\
    x_r & y_r & x_r^2 + y_r^2 & 1\\
    x_s & y_s & x_s^2 + y_s^2 & 1\\
\end{vmatrix}
\right).
\]

%==============================================================================
\clearpage
\FloatBarrier
\subsection{Scene Objects}\label{appdx:scene-objects}

The scene is composed of a collection of \emph{scene objects}, all of which must
implement the \texttt{ISceneObject} interface. This interface obligates scene
objects to have a name, color, be able to intersect itself with a 3D ray for
picking (\ref{appdx:picking}), and define what it means to be selected or
deselected.

\lstinputlisting[caption=ISceneObject Interface, 
  label=isceneobject-interface]{code-samples/isceneobject.cpp}  
 
%==============================================================================

\FloatBarrier
\subsection{Intersection Types}\label{appdx:intersection-types}

The geometry library defines a number of intersection types that represent the
intersection between two objects. It makes sense to create full types for
intersections because there is often quite a bit of information to capture (is
the intersection empty, a point, a line segment, a ray? when did the
intersection occur?). Perhaps more interestingly, intersection types nicely
align the C++ concept of constructors with our geometric concept of a 
construction. The design is inspired by David Eberly's
\href{http://www.geometrictools.com/Source/Intersection2D.html}{geometric
tools}. Intersection types are placed in the \texttt{Intersection} namespace and
follow a naming convention: simply repeat the names of the two types being
intersected. The \texttt{Line\_2rLine\_2r} type is a good example:

\lstinputlisting[float, caption=Line\_2rLine\_2r Intersection Type,
  label=intersection-line2rline2r]{code-samples/intersection-line2rline2r.cpp}

%==============================================================================

\FloatBarrier
\subsection{Picking}\label{appdx:picking}

\paragraph{The Problem.} Picking is the process of selecting the foremost object
in a view that lies under a user's mouse click. In our case, we are given an 
input mouse click from either the orthographic or perspective views, and must
choose from among the scene objects.

\paragraph{Possible Solutions.}

\begin{enumerate}
  \item Color each object a unique color, render a frame, and use the
  final pixel color to determine which object to select. OpenGL provides a
  picking mode that does this.
  \item Generate a 3D world-space ray from the mouse click and intersect this
  ray with scene objects to determine which object it intersects first. 
\end{enumerate}

\paragraph{Our Solution.} We chose method 2 because it was more straightforward
to get working than 1 and probably more efficient. In particular, we take click
events from the orthographic and perspective views and convert them into a 3D
worldspace ray. We iterate over all scene objects (there is no hierarchical
acceleration structure) and produce \texttt{Ray\_3rSceneObject} intersection
objects. These objects are quite simple, see listing
\ref{intersection-ray3rsceneobject}. We hill climb over the intersection times
to determine which object intersected first.

\lstinputlisting[caption=Ray\_3rSceneObject Intersection Type,
  label=intersection-ray3rsceneobject]{code-samples/intersection-ray3rsceneobject.cpp}

\paragraph{Selecting and Deselecting.} The \texttt{Select} method should usually
just highlight the object's edges and vertices with the global selection color
(for the moment, methods all just use \texttt{Visual::Color::SKYBLUE},
\href{https://github.com/unc-compgeom/DDAD/issues/3}{issue $\triangle$ 3}
specifies a solution.) Listing \ref{intersection-select} shows an example
\texttt{Select} implementation from \texttt{ScenePolyline\_2r}.

\lstinputlisting[float, caption=Example Select Method,
  label=intersection-select]{code-samples/intersection-select.cpp}

%==============================================================================

\FloatBarrier
\subsection{Order-independent Transparency}\label{appdx:oit}

\paragraph{The problem.} Rendering transparent surfaces is not as easy as one
might imagine. Most ways of compositing layers of translucent pixels on top of
one another show \emph{order-dependency}, meaning that the order that surfaces
are rendered can affect the final pixel color. This problem arises because the
terms in the blending equation are not commutative, so different orders of
compositing result in different pixel colors.

\paragraph{Possible solutions.} There are a number of techniques to achieve
\emph{order-independent transparency} (OIT), including
A-buffers~\cite{carpenter1984buffer}, depth
peeling~\cite{everitt2001interactive}, dual depth
peeling~\cite{bavoil2008order}, and others. Hardware manufacturers are moving
toward providing special hardware to solve OIT quickly for real-time rendering
applications. OIT techniques may be categorized into exact and approximate
methods. The ground truth is the industry-accepted OVER operator. Examples of
approximate techniques include McGuire's weighted, blended
OIT~\cite{mcguire2013weighted}.

\paragraph{Our solution.}
 
The workbench does not currently have a full implementation of OIT available,
but the functionality is specified in
\href{https://github.com/unc-compgeom/DDAD/issues/4}{issue $\triangle$ 4}.
However, the main framework is in place. Most OIT techniques require a
separation between opaque and transparent geometry. The visual interface
supports transparent materials, and when we generate vertex buffers to send to
OpenGL for rendering, opaque and transparent vertices are segregated into
different buffers. This means that it is easy to render opaque objects before
transparent ones.
